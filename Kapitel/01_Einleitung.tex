%***************************************************************************************************
%********************  EINLEITUNG  *****************************************************************
%***************************************************************************************************
\chapter[Einleitung]{\label{sec:Einleitung}Einleitung}
% max 1 Seite

Die vorliegende Dissertation widmet sich einem hochaktuellen Thema im Bereich der Energiespeicherung für die e-Mobilität \cite{Huo2015, Donateo2015,Jochem2015,Kim2014,Orsi2016,Silva2011,Holdway2010,Sternberg2015,Ramachandran2015} und mobile Robotik \cite{Hecht2023,Mikolajczyk2023,Ghobadpour2023,Mikolajczyk2023a}: der computergestützten Identifikation von verbesserten Strukturbatterien. Strukturbatterien stellen eine innovative Lösung dar, da sie über deutlich verbesserte mechanische Eigenschaften im Vergleich zu herkömmlichen Batterien verfügen und direkt in die Struktur von Fahrzeugen integriert werden können, ohne auf ein separates Gehäuse angewiesen zu sein. Diese Eigenschaft eröffnet Potenziale für eine gesteigerte Gesamtenergiedichte, selbst wenn die Strukturbatterien aufgrund ihrer Bauweise zunächst niedrigere Energiedichten aufweisen.

Durch die dringende Notwendigkeit, die Energiedichte von Batterien weiter zu steigern, um den Anforderungen an die Reichweite und Leistungsfähigkeit von Elektrofahrzeugen gerecht zu werden, ist dieses Thema so relevant wie nie. Trotz der vielversprechenden mechanischen Eigenschaften von Strukturbatterien bleiben die bisherigen Ergebnisse bezüglich ihrer elektrochemischen Leistung hinter den Erwartungen zurück.

Die bisherige Forschung hat sich hauptsächlich auf die Untersuchung einzelner Komponenten wie Strukturelektrolyte und den Bau spezifischer Konfigurationen konzentriert. Dabei wurde jedoch eine ganzheitliche Methodik zur Verknüpfung der Erkenntnisse aus den verschiedenen Teilbereichen vernachlässigt. Zudem basiert die Forschungsmethodik größtenteils auf experimentellen Ansätzen und wenigen computergestützten Modellen, die oft extrem rechenintensiv sind und oft auf den Einsatz von Supercomputern angewiesen sind.

Diese Dissertation zielt darauf ab, diese Lücke zu schließen, indem sie eine digitale Methode entwickelt, die vorhandene Erkenntnisse aus Literatur und Experimenten zu den wichtigsten Komponenten von Strukturbatterien sammelt und mithilfe von computergestützten Modellen verknüpft. Das entwickelte Framework, genannt QUINTUS, ermöglicht es, schnell und effizient Aussagen über die Leistungsfähigkeit entstehender Batteriekonfigurationen zu treffen.

Das zugrundeliegende Prinzip dieses Lösungsansatzes besteht darin, dass Computer besser dazu geeignet sind, eine Vielzahl von einfachen Zusammenhängen zu verarbeiten und wiederholt anzuwenden. Im Gegensatz zu bestehenden Ansätzen beginnt die Modellierung nicht auf atomarer Ebene, sondern auf der Komponentenebene, was eine schnellere Generierung von Ergebnissen ermöglicht. Zudem kann das Modellsystem leicht um Modelle auf mikro- oder molekularer Ebene erweitert werden, um zusätzliche Einflussfaktoren zu berücksichtigen.

%***************************************************************************************************
%********************  Motivation und Zielstellung  ************************************************
%***************************************************************************************************
\section{\label{sec:Motivation_Zielstellung}Problemstellung und Zielsetzung}
% rund 1,5 Seiten inklusive Bild

Die 

%***************************************************************************************************
%********************  LITERATURÜBERSICHT  *********************************************************
%***************************************************************************************************
\section{\label{sec:Literaturübersicht}Literaturübersicht}
% 3-5 Seiten
- Forschung nach neuen Kombinationen war lange und ist experimentell getrieben
- zunehmener Einfluss von DFT etc. aber immer nur auf niedriger Ebene

- Einzige wirkliche Vorhersagen auf dem Gebiet Strukturspeicher durch Carsted, aber nur an 3 Variaenten und ohne Elektrolytische Effekte zu Berücksichtigen 

- Jedoch zahlreiche multiphysikalische Modelle für Batterie Management Systemens BMS, die jedoch hauptsächlich für Regelungssysteme eingesetzt werden.

Die im Rahmen der vorliegenden Arbeit erarbeiteten Simulationsmodelle und numerisch ermittelten vorteilhaften Parameterkombinationen dienten \textsc{Kühn} und \textsc{Seidel-Greif} als Grundlage für ihre experimentellen Untersuchungen zum Einsatz .
Mithilfe der entwickelten Methode konnte eine optimierte  Strukturbatterie für einen hybriden Anwendungsfall identifiziert werden und zeigte durch einen ersten Funktionsprototypen eine xx\% höhere multifunktionalen Performanz gegenüber bisher veröffentlichten Strukturbatterien.









   

