\chapter{Stand der Forschung}

Im folgenden Kapitel wird ein grundlegendes Verständnis für die Funktionsweise von Strukturbatterien vermittelt werden. Außerdem werden die Besonderheiten gegenüber konventionellen Batterien oder faserverstärkten Verbundwerkstoffen erläutert. Dazu werden die wichtigsten Eigenschaften und ihre Ermittlungsverfahren erläutert und Rolle der Einzelkomponenten im Zusammenhang der Materialauswahl näher erklärt. Anschließend werden aktuelle Entwicklungsansätze diskutiert und abschließend die ungelösten Herausforderungen mit den aktuellen Methoden näher analysiert.

\section{Grundlagen der Strukturbatterie}
Strukturbatterien sind 

\section{Wichtigsten Eigenschaften und ihre Ermittlungsverfahren}
%\subsection{Interkalation}
\subsection{Gesamtkapazität}
\subsection{Energiedichte}
\subsection{Elektrische Spannung}
\subsection{Zyklenverhalten}
\subsection{Steifigkeit und Festigkeit}
%\subsection{Mechanische Spannung}
\subsection{Multifunktionale Effizienz}

\section{Materialauswahl}
\subsection{Anode}
\subsection{Kathode}
\subsection{Elektrolyte}
\subsection{Separator}


\section{Aktuelle Ansätze zur Entwicklung und Auslegung von Strukturbatterien}

\section{Ungelöste Herausforderungen in der Entwicklung von Strukturbatterien}
Erstellung Bild siehe Kommentar in .tex Datei
%Bild in Inkscape erzeugt und als SVG sowie pdf_tex speichern (Speichern unter -> .pdf -> Text in PDF weglassen und LaTex Datei erstellen). 


\begin{figure}[h]
	%\raggedleft
		%\def\svgwidth{\columnwidth}
	\def\svgscale{0.98}
		\input{testbild.pdf_tex} 
		\caption{\label{fig:testbild}Testbild erzeugt mit Inkscape}
\end{figure}

Bild \ref{fig:testbild} \cite{Dannemann.Kucher_et.al_AppliedSciences_2018}  

Verwendung Package SIUNITX %(siehe Datei latex_package_readme_siunitx.pdf)   

Anzugsdrehmoment von $M_{\textnormal{a}}=\SI{1.1}{\newtonmetre}$

von \SI{1}{\kilo\hertz} bis \SI{15}{\kilo\hertz}

mittlere Temperaturänderung von $\left\langle \Delta T_{\textnormal{p}}\right\rangle(t)<\SI{0.5}{\degreeCelsius}$

Masse von $m_{\textnormal{p}}=\SI[separate-uncertainty]{0.884 (15)}{\gram}$

(siehe Abschnitt \ref{ch:anhang})