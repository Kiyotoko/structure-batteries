%\chapter*{Symbolverzeichnis}
%\label{sec:Symbolverzeichnis}
\addchap{Symbolverzeichnis}
\markboth{Symbolverzeichnis}{Symbolverzeichnis}


%Einzug erste Spalte
\newlength{\TabulatorVZ} % definiert einen neuen L�ngenparameter
\settowidth{\TabulatorVZ}{$m$, $n$, $i$, $\imath$, ${\bar{\imath}}$, $\jmath$\quad} % Setzt den L�ngenparameter auf den Wert, den der Text hat
% Einzug Einheitenspalte
\newlength{\TabulatorEH} % definiert einen neuen L�ngenparameter
\setlength{\TabulatorEH}{\widthof{$\si{\celsius}$; $\si{\kelvin}$}}

\newlength{\TabulatorTX}
\setlength{\TabulatorTX}{\textwidth}
\addtolength{\TabulatorTX}{-\TabulatorVZ-\TabulatorEH-2\tabcolsep}

{\renewcommand*{\arraystretch}{1.2}%

\section*{Abkürzungen}

\begin{longtable}{@{}p{\TabulatorVZ}@{}p{\TabulatorTX+\TabulatorEH+2\tabcolsep}@{}}
FE								& Finite Elemente \\
FEM								& Finite"=Elemente"=Methode \\
PA/PA6						& Polyamid/Polyamid-6 \\
PEEK    	        & Polyetheretherketon \\
PP								& Polypropylen

\end{longtable}

\section*{Allgemeine Notation}

\begin{longtable}{@{}p{\TabulatorVZ}@{}p{\TabulatorTX+\TabulatorEH+2\tabcolsep}@{}}
a									& Skalar \\
\textbf{a}				& Tensor 1. Stufe (Vektor)
\end{longtable}

\section*{Lateinische Buchstaben}

\begin{longtable}{@{}p{\TabulatorVZ}@{}p{\TabulatorTX}p{\TabulatorEH}@{}}
	$A$					& Fläche					& $\si{\metre\squared}$ \\
	$C$					& Capazität					& $\si{\ampere\s}$					\\
	$c$					& Konzentration				& $\si{\mole\per\metre\cubed}$\\
	$D$					& Diffusionskonstante		& $\si{\metre\squared\per\second}$\\
	$E$					& Elastizitätsmodul			& $\si{\pascal}$	\\
	$F$					& Kraft						& $\si{\newton}$	\\
	$F_{\text{K}}$		& Faraday-Konstante			& $\si{\coulomb\per\mole}$	\\
	$f_{\pm}$			& Aktivitätskoeffizient		& $\si{\coulomb\per\mole}$	\\
	$G$					& Schubmodul				& $\si{\pascal}$	\\
	$h$					& Plattendicke				& $\si{\metre}$		\\
	$\textbf{i}$		& Stromdichte				& $\si{\ampere\per\metre\squared}$		\\
	$j$					& molare Ionenflussdichte	& $\si{\mole\per\metre\squared\per\second}$		\\
	$K$					& Konstante					& -					\\
	$R_{\text{K}}$		& Unverselle-Gaskonstante	& $\si{\joule\per\mole\per\kelvin}$	\\
	$U_{\theta}$ 		& Elektrochemisches Standardpotenzial	& $\si{\volt}$ \\
	$V$					& Volumen					& $\si{\cubic\metre}$
\end{longtable}

\section*{Griechische Buchstaben}
\begin{longtable}{@{}p{\TabulatorVZ}@{}p{\TabulatorTX}p{\TabulatorEH}@{}}
	$\varepsilon$			& Dehnung						& -	\\
	$\vartheta$				& Temperatur					& $\si{\celsius}$; $\si{\kelvin}$ \\
	$\sigma$				& elektrische Leitfähigkeit		& $\si{\pascal}$ \\
	$\boldsymbol{\sigma}$	& Mechanischer Spannungstensor	& $\si{\pascal}$ \\
	$\sigma_{\text{B,K}}$	& Boltzmann"=Konstante 			& $\si{\joule\per\kelvin}$ \\
	$\nu$					& Poissonzahl					& -	\\
	$\rho$					& Dichte						& $\si{\kilo\per\metre\cubed}$
\end{longtable}

\section*{Indizes, Exponenten und mathematische Akzente}

\begin{longtable}{@{}p{\TabulatorVZ}@{}p{\TabulatorTX+\TabulatorEH+2\tabcolsep}@{}}
	$x_{\text{AM}}$				& Aktivmaterial\\
	$x_{\text{b}}$				& Binder\\
	$x_{\text{DS}}$				& Deckschicht\\
	$x_{\text{echem}}$			& elektro"=chemisch\\
	$x_{\text{exp}}$			& experimentell\\
	$x_{\text{l}}$				& leitende Phase\\
	$x_{\text{s}}$				& specihernde Phase\\
	$x_{\text{mech}}$			& mechanisch\\
	$x_n$						& Normal zur Oberfläche	\\
	$x_{-}$						& negative Elektrode \\
	$x_{\text{OCV}}$			& Gleichgewichtsspannung (\textit{engl.} open-circuit voltage) \\
	$x^{+}$						& positive Elektrode \\
	$x_{\text{th}}$				& thermisch \\
	$x_{\text{theor}}$			& theoretisch \\
	$\vec{x}^T$					& Transponierter Vektor	\\
	$\tilde{x}$				    & Effektivwert 	\\
	$\avrg{x}$					& Mittelwert
\end{longtable}

} % Ende arraystrecth


\clearpage