%\chapter*{Symbolverzeichnis}
%\label{sec:Symbolverzeichnis}
\addchap{Symbolverzeichnis}
\markboth{Symbolverzeichnis}{Symbolverzeichnis}


%Einzug erste Spalte
\newlength{\TabulatorVZ} % definiert einen neuen L�ngenparameter
\settowidth{\TabulatorVZ}{$m$, $n$, $i$, $\imath$, ${\bar{\imath}}$, $\jmath$\quad} % Setzt den L�ngenparameter auf den Wert, den der Text hat
% Einzug Einheitenspalte
\newlength{\TabulatorEH} % definiert einen neuen L�ngenparameter
\setlength{\TabulatorEH}{\widthof{$\si{\celsius}$; $\si{\kelvin}$}}

\newlength{\TabulatorTX}
\setlength{\TabulatorTX}{\textwidth}
\addtolength{\TabulatorTX}{-\TabulatorVZ-\TabulatorEH-2\tabcolsep}

{\renewcommand*{\arraystretch}{1.2}%

\section*{Abk�rzungen}

\begin{longtable}{@{}p{\TabulatorVZ}@{}p{\TabulatorTX+\TabulatorEH+2\tabcolsep}@{}}
FE								& Finite Elemente \\
FEM								& Finite"=Elemente"=Methode \\
PA/PA6						& Polyamid/Polyamid-6 \\
PEEK    	        & Polyetheretherketon \\
PP								& Polypropylen

\end{longtable}

\section*{Allgemeine Notation}

\begin{longtable}{@{}p{\TabulatorVZ}@{}p{\TabulatorTX+\TabulatorEH+2\tabcolsep}@{}}
a									& Skalar \\
\textbf{a}				& Tensor 1. Stufe (Vektor)
\end{longtable}

\section*{Lateinische Buchstaben}

\begin{longtable}{@{}p{\TabulatorVZ}@{}p{\TabulatorTX}p{\TabulatorEH}@{}}
	$A$										& Fl�che																										& $\si{\metre\squared}$ \\
	$C$										& Konstante																								& -					\\
	$E$										& Elastizit�tsmodul																				& $\si{\pascal}$	\\
	$F$										& Kraft																										& $\si{\newton}$	\\
	$G$										& Schubmodul																								& $\si{\pascal}$	\\
	$h$                   & Plattendicke																							& $\si{\metre}$		\\
	$V$										& Volumen																									& $\si{\cubic\metre}$
\end{longtable}

\section*{Griechische Buchstaben}
\begin{longtable}{@{}p{\TabulatorVZ}@{}p{\TabulatorTX}p{\TabulatorEH}@{}}
	$\varepsilon$					& Dehnung																									& -					\\
	$\vartheta$						& Temperatur																								& $\si{\celsius}$; $\si{\kelvin}$	\\
	$\sigma$							& Spannung																									& $\si{\pascal}$
\end{longtable}

\section*{Indizes, Exponenten und mathematische Akzente}

\begin{longtable}{@{}p{\TabulatorVZ}@{}p{\TabulatorTX+\TabulatorEH+2\tabcolsep}@{}}
$x_{\text{DS}}$				& Deckschicht																							\\
$x_n$									& Normal zur Oberfl�che																		\\
$\vec{x}^T$						& Transponierter Vektor																		\\
$\tilde{x}$				    & Effektivwert 																						\\
$\avrg{x}$						& Mittelwert
\end{longtable}

} % Ende arraystrecth


\clearpage