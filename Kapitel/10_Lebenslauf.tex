\pagestyle{empty}
\section*{Lebenslauf}

\begin{tabbing}
\hspace*{4cm}\= \kill

\textbf{Persönliche Daten}\\[0.45em]%
Name							\>  Willi Zschiebsch	\\[0.45em]
Geburtsdatum					\> 	28.10.1996			\\[0.45em]
Geburtsort						\> 	Wurzen  			\\[0.45em]
Eltern							\> 						\\
								\> 						\\[0.45em]
Familienstand					\> 	ledig				\\[0.45em]
Kinder      					\> 	keine				\\[0.45em]
            					\> 						\\[0.45em]

Staatsangehörigkeit		\> 		deutsch					\\[1.5em]

\textbf{Ausbildung} \\[0.45em]
2015 - 2018	\> Maschinenbau Master an der HTWK Leipzig	\\[0.5em]	
2018	\> Forschungsaufenthalt im Robotiklabor von Prof. Amir Shapiro 	\\[0.5em]
2015 - 2018	\> Maschinenbau Bachelor an der HTWK Leipzig	\\[0.5em]
2015	\> Abitur Wilhelm-Ostwaldgymnasium Leipzig	\\[0.5em]
															\\[1.5em]

\textbf{Auszeichnungen und Stipendien} \\[0.45em]
03/2015 \> Sonderpreis des Ministeriums für Wirtschaft, Arbeit und Verkehr, Sachsen \\[0.5em]
06/2015 \>  Preis für eine besondere Leistung im Gebiet der Technik von Heinz und \\ 
		\> Gisela Friederichs Stiftung \\[0.5em]
06/2015 \>  Qualifizierung und Teilnahme für den Bundeswettbewerbes „jugendForscht“ \\[0.5em]
12/2015–12/2020 \>  Mitglied in der Studienstiftung des deutschen Volkes \\[0.5em]
10/2018 \> Preis der Karl-Kolle-Stiftung \\[0.5em]
06/2019 \> Dritter Platz des VDI-Förderpreises \\[0.5em]
10/2021 \> Preis der Karl-Kolle-Stiftung \\[0.5em]
06/2021-05/2024 \>  HTWK-Promotionsstipendium \\[0.5em]
04/2022 \> Best Speaker-Award auf der ICOMS21 \\[0.5em]
09/2022 \> Best Presentation Award auf der ICAMDS 2023 \\[0.5em]

\end{tabbing}





%%Diplom-\\
%%praktikum:\> 1994\> Institut f"ur Theoretische Physik\\
%%\>\>(Abt. B) der TU Clausthal\\
%wiss. Hilfskraft:\> 1993 - 1994\> Physikalisches Institut (Abt. Experimentalphysik)\\
%\>\> der TU Clausthal\\
%\> 1995\>  Institut f"ur Theoretische Physik\\
%\>\> (Abt. B) der TU Clausthal\\
%\> 1996 - 1998\> Institut f"ur Leichtbau und Kunststofftechnik\\
%\>\> der TU Dresden\\
%Stipendien:\> 1997 - 1998\> Landesinnovationsstipendium Sachsen (LIST)\\
%wiss. Mitarbeiter:\> seit 1998\> Institut f"ur Leichtbau und Kunststofftechnik\\
%\>\> der TU Dresden
%\end{tabbing}