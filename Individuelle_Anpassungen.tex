\typeout{Anpassungen von mda, Stand 10.10.2012 (teilweise basierend auf TEXDEFS.TEX)}

\newboolean{Entwurfsmodus}
\setboolean{Entwurfsmodus}{true} % true|false

\definecolor{lightgray}{rgb}{0.5,0.5,0.5}
\definecolor{lightgrey}{rgb}{0.5,0.5,0.5}
\definecolor{darkred}{rgb}{0.6,0,0}

%\newcommand{\avg}[1]{\left< #1 \right>} % for average
\newcommand{\rot}[1]{\textcolor{red}{#1}} %rote Hervorhebung

% Breitenangabe fuer Strukturgramme
% 	Es ist leider nicht moeglich die Breite eines Strukturgramms auf die Textbreite zu beziehen, daher 150 fuer A4, und 105 fuer A5 einstellen
%		sProofOn muss trotzdem noch fuer jedes Diagramm wegen der Hoeheneinstellung durchgefuehrt werden!!!
\newcommand{\struktbreite}{150}

% Fuer eigene Verwendung
\ifthenelse{\boolean{Entwurfsmodus}}
{
\typeout{Entwurfsmodus an.}
\newcommand{\korr}[1]{\textbf{\textcolor{magenta}{#1}}} %rote Hervorhebung
\newcommand{\kommentar}[1]{ \textcolor{green}{\textit{(#1)}} } % Kommentar; in Endfassung einfach Inhalt der geschweiften Klammer loeschen
\newcommand{\fertig}{\texorpdfstring{ \textcolor[rgb]{0,0.8,0}{(\checkmark)}}{ o.k.}} % macht ein Haekchen innnherhalb von Ueberschriften
\newcommand{\inArbeit}{\texorpdfstring{ \textcolor[rgb]{0.8,0,0}{(\checkmark)}}{ i.A.}} % macht ein Haekchen innnherhalb von Ueberschriften
\newcommand{\fehlt}{\texorpdfstring{ \textcolor[rgb]{0.8,0,0}{(\textbf{!!})}}{ fehlt}} % macht ein Haekchen innnherhalb von Ueberschriften
\newcommand{\alt}[1]{\underline{ALT:} \textcolor{lightgray}{#1}} %alte Textpassagen
\newcommand{\neu}[1]{\underline{NEU:} \textcolor{darkred}{#1}} %neue Textpassagen
}{
%%%%% Fuer Ausdruck und Weitergabe
\typeout{Entwurfsmodus aus.}
\newcommand{\korr}[1]{} %rote Hervorhebung
\newcommand{\kommentar}[1]{} % Kommentar; in Endfassung einfach Inhalt der geschweiften Klammer loeschen
\newcommand{\fertig}{} % macht ein Haekchen innnherhalb von Ueberschriften
\newcommand{\inArbeit}{} % macht ein Haekchen innnherhalb von Ueberschriften
\newcommand{\fehlt}{} % macht ein Haekchen innnherhalb von Ueberschriften
\newcommand{\alt}[1]{} %alte Textpassagen
\newcommand{\neu}[1]{#1} %neue Textpassagen
}

%% Sonstige Definitionen
% Original war hinter Bild Tabelle Abschnitt immer ein ~ - also ein nicht trennbares Leerzeichen
\newcommand{\bild}[1]{Bild \ref{#1}} % Bildverweise
\newcommand{\kreis}[1]{\unitlength1ex\begin{picture}(2.5,2.5) \put(0.75,0.75){\circle{2.5}}\put(0.75,0.75){\makebox(0,0){#1}}\end{picture}} % Einekreiste Zahl oder Buchstabe
\newcommand{\tabelle}[1]{Tabelle \ref{#1}} % Tabellenverweise
\newcommand{\abschnitt}[1]{Abschnitt \ref{#1}} % Gliederungsverweise
\newcommand{\anhang}[1]{Anhang \ref{#1}} % Verweise zum Anhang
\newcommand{\name}[1]{\textsc{#1}} % Namentliche Nennungen
\newcommand{\fett}[1]{\textbf{#1}} % Fett
\newcommand{\ul}[1]{\underline{#1}} % unterstrichen --> in Paket soul bereits definiert
\newcommand{\dul}[1]{\underline{\underline{#1}}} % Doppelt unterstrichen
\newcommand{\mat}[1]{\pmb{#1}} % Kennzeichnung von Matrizen besser als \pmb (poor mans bold) waere \boldsymbol, welches allerdings leider nicht funktioniert -> fehlende Schriftart???
\newcommand{\kmplx}[1]{\ul{#1}} % Kennzeichnung komplexer Groessen
\newcommand{\kkmplx}[1]{\kmplx{#1}^\ast} % Kennzeichnung einer konjungiert komplexen Groesse
\newcommand{\oampl}[1]{\hat{#1}} % Ortsamplitude
\newcommand{\zampl}[1]{\check{#1}} % Zeitamplitude
\newcommand{\avrg}[1]{\overline{#1}} % Mittelwert
\newcommand{\gl}[1]{(\ref{#1})} % Gleichungsverweis
\newcommand{\UD}{\text{UD}}	% UD in Formeln (unidirektional)
\newcommand{\TV}{\text{TV}} % TV in Formeln (textilverstaerkt)
\newcommand{\EX}{\text{Exp.}} % Exp. in Formeln (experimentell ermittelt)
\newcommand{\FE}{\text{FE}} % FE in Formeln
\newcommand{\DS}{\text{DS}} % Deckschicht DS in Formeln
\newcommand{\KS}{\text{KS}} % Kernschicht DS in Formeln
\newcommand{\MAC}{\text{MAC}} % MAC in Formeln
\newcommand{\Terz}{\text{Terz}} % MAC in Formeln
\newcommand{\Wert}{\text{Wert}} % Formelzeichen Wert
\newcommand{\Geni}{{\text{Gen}_i}} % Gen_i in Formeln
\newcommand{\Kind}{\text{Kind}} % Kind in Formeln
\newcommand{\Elter}{\text{Elter}} % Elter in Formeln
\newcommand{\Lag}{\text{Lag}} % Lag fuer Lagerung in Formeln
\newcommand{\Mat}{\text{Mat}} % Mat fuer Material in Formeln
\newcommand{\DEVOP}{\name{DevOP}\xspace} % DEVOP- im Text

%-------------------------------------------------------------------
% Globale Trennvorschlaege
\hyphenation{Donau-dampf-schiff-fahrt}
\hyphenation{Ur-instinkt}
\hyphenation{ei-nen}
\hyphenation{vi-bro-akus-tisch}
\hyphenation{vi-bro-akus-tische}
\hyphenation{vi-bro-akus-tisch-en}
\hyphenation{vi-bro-akus-tisch-em}
\hyphenation{vi-bro-akus-tisch-es}
\hyphenation{vi-bro-akus-tisch-er}
\hyphenation{Vi-bro-akus-tik}
\hyphenation{Mehr-schicht-ver-bund}
\hyphenation{Straf-funk-tion-en}
\hyphenation{an-iso-trop}
\hyphenation{an-iso-tro-pe}
\hyphenation{an-iso-tro-pen}
\hyphenation{an-iso-tro-pem}
\hyphenation{an-iso-tro-per}
\hyphenation{An-iso-tro-pie}
\hyphenation{uni-di-rek-tio-nal}
\hyphenation{uni-di-rek-tio-nale}
\hyphenation{uni-di-rek-tio-nal-en}
\hyphenation{Leicht-bau-struk-tur-en}
\hyphenation{vis-ko-elas-tisch}
\hyphenation{be-an-spruch-ten}
\hyphenation{Stich-pro-ben-grö-ße}
\hyphenation{me-si-o-buk-kal}
\hyphenation{pa-la-tal}
\hyphenation{ste-re-o-mi-kros-ko-pi-sche}

%-------------------------------------------------------------------
%% Ligaturen - leider nicht global einstellbar
%% ff fi fl ffi ffl ll lll
% Schall"|leistung
% werkstoff"|inhärent
% Werkstoff"|integration


%-------------------------------------------------------------------
%\newcommand{\entspr}{\widehat{=}}
\newcommand{\entspr}{\stackrel{\scriptscriptstyle\wedge}{=}}
\newcommand*{\dunderl}[1]{\underline{\underline{#1\!}}\,}
\renewcommand{\Re}{\mbox{Re}} %normalerweise: Schn"orkel-R
\renewcommand{\Im}{\mbox{Im}} %normalerweise: Schn"orkel-I
\renewcommand{\i}{{\rm i}}    %normalerweise: i ohne Punkt
\newcommand{\de}{{\mathsf d}}  %sans serif für elektrische Groessen
\newcommand{\e}{{\mathsf e}}
\newcommand{\E}{{\mathsf E}}
\newcommand{\U}{{\mathsf U}}
\newcommand{\T}{{\rm T}} % Transponiert-Operator
\newcommand{\N}{\mbox{I\hspace{-0.19em}N}} % Menge N = nat"urliche Zahlen
\newcommand{\R}{\mbox{I\hspace{-0.19em}R}} % Menge R = reelle Zahlen
\newcommand{\C}{\mbox{\hspace{0.35em}\rule{0.04em}{1.55ex}\hspace{-0.35em}C}} % Menge C = komplexe Zahlen
\newcommand{\mitDelta}{\mathnormal{\Delta}}  % neu: LaTeX 2e
\newcommand{\mitTheta}{\mathnormal{\Theta}}
\newcommand{\mitPhi}{\mathnormal{\Phi}}
\newcommand{\mitPsi}{\mathnormal{\Psi}}
\newcommand{\brmitTheta}{\breve{\mitTheta}}
\newcommand{\brmitPhi}{\breve{\mitPhi}}
\newcommand{\brmitPsi}{\breve{\mitPsi}}
\newcommand{\brrho}{\breve{\rho}}
% fuer Buchstaben mit Kreisen
\newcommand*\mycirc[1]{%
\begin{tikzpicture}[baseline=(C.base)]
\node[draw,circle,inner sep=1pt,minimum size=3ex](C) {#1};
\end{tikzpicture}}

%\renewcommand{\theta}{\fehlertheta}      % \
%\renewcommand{\theta}{\vartheta}
\renewcommand{\phi}{\varphi}
%\renewcommand{\phi}{\fehlerphi}          %  zur Sicherheit
%\renewcommand{\epsilon}{\textepsilon}  % /
%\renewcommand{\epsilon}{\fehlerepsilon}  % /
% Das Symbol \phi (hier \varvarphi) wird fuer Sonderfaelle benoetigt.
% Folgende Definition entstammt fontmath.ltx.
\DeclareMathSymbol{\varvarphi}{\mathord}{letters}{"1E}
\DeclareMathSymbol{\eps}{\mathsf}{letters}{"0F} 
    
% Neue Spaltenarten für Tabellen
\newcolumntype{C}[1]{>{\centering\arraybackslash}p{#1}} % Mehrzeilige horizontal zentrierte Tabellenspalten
\newcolumntype{M}[1]{>{\centering\arraybackslash}m{#1}} % Mehrzeilige horizontal und vertikal zentrierte Tabellenspalten
%\newcolumntype{V}[1]{>{\RaggedRight\arraybackslash}m{#1}} % Mehrzeilige linksbündige und vertikal zentrierte Tabellenspalten
\newcolumntype{L}[1]{>{\RaggedRight\arraybackslash}p{#1}} % Mehrzeilig linksbündige oben ausgerichtete Spalten
\newcolumntype{R}[1]{>{\RaggedLeft\arraybackslash}p{#1}} % Mehrzeilig linksbündige oben ausgerichtete Spalten
\newcolumntype{d}{D{.}{,}{-1}}													% Ausrichtung am Dezimalpunkt und Satz als Komma
\newcommand{\tabkopf}[1]{\textbf{#1}}
\newcommand{\Dkopf}[1]{\multicolumn{1}{c}{\textbf{#1}}}% tabkopf für Spaltentyp D
\newcommand{\DkopfZU}[2]{\multicolumn{1}{M{#1}}{\textbf{#2}}} % tabkopf mit Zeilenumbruch
\newcommand{\DZU}[2]{\multicolumn{1}{M{#1}}{#2}} % tabkopf mit Zeilenumbruch
\newcommand{\Dtext}[1]{\multicolumn{1}{c}{#1}} 				% normaler Text im Spaltentyp D
\newcommand{\DSkopf}[2]{\multicolumn{1}{M{#1}|}{\textbf{#2}}}% tabkopf für Spaltentyp S mit Umbruchmöglichkeit
\newcommand{\DStext}[2]{\multicolumn{1}{M{#1}|}{#2}} 				% normaler Text im Spaltentyp S mit Umbruchmöglichkeit
% Einfacher Abstand in Tabellenumgebung mtabular
\newenvironment{mtabular}{%
	\small%
  \renewcommand*{\arraystretch}{1.4}%
  \tabular
}{%
  \endtabular
}

\pdfsuppresswarningpagegroup=1 %mk Unterdrueckung warning of multiple PDF files

\usepackage{lscape} %mk Tabelle Querformat
\usepackage{amsmath} %mk for boldsymbol

\usepackage[percent]{overpic} %mk Bild auf ganzer Seite