\begin{table}[ht]
    \centering
    \caption{Übersicht bisher entwickelter Strukturbatterien.}
    \begin{tabular}[t]{lcccc}
    \toprule
    &\makecell{Kapazität\\$\left[ \si{\mA \hour \per \g} \right]$} % \textsuperscript{*}
    &\makecell{E-Modul\\ $\left[ \si{\GPa} \right]$}
    &\makecell{Zugfestigkeit\\ $\left[ \si{\MPa} \right]$}
    &\makecell{Leitfähigkeit\\ $\left[ \si{\siemens \per \cm} \right]$}
    %&CR [\%] % Capacity Retention
    %&$\text{D}_{\text{Li}}$ %[$\text{cm^2/s}$]
    %&Ref.
    \\
    \midrule
    Grafit
        &356-372 \cite{Winter1998} % Capacity
        &10 \cite{Lin2023} % E-Module
        &31,29 \cite{Lin2023} % Zugfestigkeit
        &$10^3-10^4$ \cite{Wang2021} % Leitfähigkeit
        %&98
        %&$10^{-7}-10^{-6}$ ($10^{-11}$\textsuperscript{,K})
        %&\cite{Persson2010,Wang2021,Olutogun2024}\\
        \\
    Graphen
        &770/1115 \cite{Wu2011} % Capacity
        &31,29 \cite{Lin2023}  % E-Module
        &130 \cite{Lin2023} % Zugfestigkeit
        &2700 \cite{Murata2019} % Leitfähigkeit
        %&100
        %&90
        %&$7 \times 10^{-5}$
        %&\cite{Zhu2014,Wang2017,Kuehne2017}\\
        \\
    CNT
        &400-600 \cite{Boaretto2020}
        &34,86 \cite{Kim2017}
        &850 \cite{Kim2017}
        &5000 \cite{Charlier2007}
        \\
    %Kohlenstofff Nanoröhren
    %    &1115
    %    &90
    %    &$10^{-14}-10^{-11}$
    %    %&\cite{Maurin1999,Zhao2000,Meunier2002,Shin2002,Nishidate2005,Schauerman2012}\\
    %Harter Kohlenstoff
    %    &200-600 % 0.2C
    %    %802-1063 lade capacitität
    %    % 27.9-47.3 lade/entlade effizienz / Columbic Efficiency
    %    &72-90 % nach 50 Zyklen
    %    &$10^{-9}$-$10^{-8}$
    %    %&\cite{Fujimoto2010,Bridges2012,Yang2012}\\
    %Karbon Aerogel
    %    &349-570,2
    %    &31,9-97%(836.9-570.2)/836.9
    %    &n.a.
    %    %&\cite{Yang2015,Pham2024,Li2022a}\\
    T300
        &130 \cite{Kjell2011}
        &230 \cite{Kjell2011}
        &3530 \cite{Kjell2011}
        &666,67\cite{Kjell2011}
        %&91
        %&46,5 % (170-91)/170
        %&$10^-12-10^-11$
        %&\cite{Uchida1996,Kjell2011,Johansen2022}
        \\
    %T300 unbeschichtet
    %    &130
    %    &62,9 %(350-130)/350
    %    &$10^-12-10^-11$
    %    &\cite{Uchida1996,Kjell2011,Johansen2022}\\
    %T800
    %    &98
    %    &42,4 % (170-98)/170
    %    &n.a.
    %    &\cite{Kjell2011,Johansen2022,Johansen2024}\\
    %T800 unbeschichtet
    %    &112
    %    &42,3 %(194-112)/194
    %    &n.a.
    %    &\cite{Kjell2011,Johansen2022,Johansen2024}\\
    IMS65
        &130 \cite{Kjell2011}
        &294 \cite{Kjell2011}
        &6000 \cite{Kjell2011}
        &689,66\cite{Kjell2011}
        %&108
        %&34,9 %(166-108)/166
        %&$10^{-8}-10^{-6}$
        %&\cite{Kjell2011}
        \\
    UMS45
        &33 \cite{Kjell2011}
        &430 \cite{Kjell2011}
        &4500 \cite{Kjell2011}
        &1030,93 \cite{Kjell2011}
        \\
    %IMS65
    %    &177
    %    &52,3 %(360-177)/350
    %    & $10^{-8}-10^{-6}$
    %    &\cite{Kjell2011,Kjell2013}\\
    \bottomrule
    \end{tabular}
    %\noindent{\footnotesize{\textsuperscript{*} Die Abkürzung nicht auffindbar (n.a.) wurde benutzt.}}
\end{table}%