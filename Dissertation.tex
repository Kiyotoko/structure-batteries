%**************************************************************************
%**************************************************************************
%*********** DOKUMENTDEKLARATION und GRUNDFORMATIERUNG ********************
%**************************************************************************
%**************************************************************************
\documentclass[
draft=false, % Entwurfsmodus an/aus true|false
DIV=14, % DIVzahl|DIV=calc - Satzspiegel (Seitenränder), ggf. geometry-Paket nutzen
fontsize=12pt, % Schriftgröße
a4paper, % Papierformat a5paper|a4paper
twoside, % Ein- oder zweiseitig oneside|twoside
numbers=noendperiod,
numbers=noenddot, % kein Punkt nach Abschnitten (statt 3.2.1. also 3.2.1)
toc=bibliography, % Literaturverzeichnis ins Inhaltsverzeichnis aufnehmen
parskip=half, %Absatzformat ohne Erstzeileneinzug, Abstand zwischen Absätzen
headings=normal, % Größe der Überschriften small|normal|big
%ngerman, % Neue Deutsche Rechtschreibung
open=any, % any|right - Kapitel fangen auf einer neuen Seite (any) an oder immer reechts (right)
headinclude=true, % Kopfzeile ist Teil des Satzspiegels
footinclude=false, % Fußzeile ist nicht Teil des Satzspiegels
BCOR=10mm, % Breite des Bundsteg
pagesize=auto, % Ausgabetreiber auto|pdftex|dvips|dvipdfmx|false|automedia
]{scrreprt} % [bibgerm] % report oder KOMA: scrreprt|scrbook

\usepackage[
headsepline, % Linie zwischen Kopfzeile und Text %
%footsepline,
autooneside, % 
automark, %
]{scrlayer-scrpage} % Für die Formatierung von Kopfzeilen

%\usepackage{showframe} % Zeigt den Satzspiegel an
%\usepackage{layouts}

\usepackage[ngerman]{babel}

%\usepackage{tocloft}
\usepackage{scrhack} % Verbessert das Zusammenspiel zwischen KOMA und dem float-Paket

\usepackage[utf8]{inputenc} %Korrekte Codierung Umlaute etc.
\usepackage{xspace}

\usepackage{tabularx}
\usepackage{multirow, makecell}
\usepackage{booktabs}


\usepackage[T1]{fontenc} %Nur ändern, wenn kyrillische Gedichte oder japanische Kanji zu setzen sind...
%%%\usepackage{geometry} % Zum manuellen Einstellen der Seitenränder etc.
%%%\geometry{
%%%twoside=false, % für zweiseitiges Layout
%%%includehead, % Kopfzeile ist Teil des Satzspiegels
%%%%includefoot, % Kopfzeile ist Teil des Satzspiegels
%%%inner=24mm,
%%%outer=24mm,
%%%top=24mm,
%%%bottom=35mm,
%%%bindingoffset=10mm, % Breite des Bundsteg
%%%}

%**************************************************************************
%**************************************************************************
%*********** PAKETDEKLARATION *********************************************
%**************************************************************************
%**************************************************************************

%------ MK global path for input -----------------------------------
\makeatletter
\def\input@path{
{Abbildungen/01_Einleitung/}
{Abbildungen/02_SoA/}
{Abbildungen/03_Modellierung/}
{Abbildungen/04_Digitalisierung/}
{Abbildungen/05_Ergebnisse/}
{Abbildungen/06_Leichtbau/}
{Abbildungen/06_Zusammenfassung/}
{Abbildungen/09_Anhang/}}
%or: \def\input@path{{/path/to/folder/}{/path/to/other/folder/}}
\makeatother

%------ MikTeX Standardpakete --------------------------------------
\usepackage[pdftex]{graphicx} \DeclareGraphicsExtensions{.pdf,.jpg}
\graphicspath{
{Abbildungen/01_Einleitung/}
{Abbildungen/02_SoA/}
{Abbildungen/03_Modellierung/}
{Abbildungen/04_Digitalisierung/}
{Abbildungen/05_Ergebnisse/}
{Abbildungen/06_Leichtbau/}
{Abbildungen/07_Zusammenfassung/}
{Abbildungen/09_Anhang/}
} % Definiere Grafikpfade

\usepackage{lmodern}

\usepackage{latexsym}     % Definiert einige Latex-Sonderzeichen (Package ist Teil von LaTeX 2e)
\usepackage{bibgerm}      %bibgerm:	Deutsches Literaturverzeichnis

\usepackage{multirow}          % enthält die Funktion eine Tabellenzelle über mehrer Zeilen zu verbinden

%----- persönl. Anpassungen ----------------------------------------
\usepackage{diss_style}				% Anpassungen mda

\usepackage{paralist}					 % für kompakte Aufzählungen (compactitem oder compactenum)
\usepackage{color}						 % für farbigen Text
\usepackage{threeparttable}		 % für Tabellen mit z.B. Fußnoten etc.
\usepackage{ragged2e}					 % für \RaggedRight u.ä. mit Silbentrennung
\usepackage{slashbox}					 % Diagonale Linien in Tabellenzelle
\usepackage{calc, dcolumn}		 % für Ausrichtung am Dezimalpunkt in Tabellen
\usepackage{amsmath}
\usepackage{amssymb}
\usepackage{amsfonts}
\usepackage{nicefrac} % Darstellung von schrägen Bruchstrichen, z.B. E'/E'' im Fließtext
\usepackage{float}
\usepackage[section]{placeins} % z.B. \FloatBarrier-Befehl zur Ausgabe aller Gleitobjekte ohne Seitenumbruch (Alternative zu \clearpage)
															% Option [section] erzwingt nach jeder Section einen \FloatBarrier-Befehl, d.h. keine Float-Objekte dürfen in die nächste Section "rutschen"
\usepackage{longtable}			% Tabellen über mehrere Seiten (z.B. beim Symbolverzeichnis)
\usepackage{textcomp} % mk \textmu
\usepackage{rotating} % mk
\usepackage{pdfpages} % mk: Einbinden von PDF-Dokumenten
\usepackage{listings} % mk: EInbindung Matlab typeset
% Für das siunitx müssen die aktuellen Pakete l3kernel und l3packages über den MikTex-Package Manager installiert werden!
\usepackage{siunitx} % für Einheiten [version-1-compatibility]
\sisetup{
list-units = single,
range-phrase = \,\ldots,
exponent-product=\cdot,
%per-mode=fraction,
output-decimal-marker={,},
detect-family,
detect-all, % Detektiert die aktuelle Schriftart
%text-rm=\normalfont, % wenn SI{} im Text auftauch als normalen Text setzen
%binary-units = true
} 
\DeclareSIUnit{\wtpercent}{wt\%}
\DeclareSIUnit{\newtonmetre}{Nm}
\DeclareSIUnit{\tanDeltaMilli}{10^{-3}}

\usepackage{hvfloat} % Für Querformatobjekte, die nicht übner die ganze Seite gehen

%\usepackage[color]{changebar} % Für seitliche Striche, insb. zur Hervorhebung von Änderungen
%	\cbcolor{red}
%	\setlength{\changebarwidth}{5pt}

\usepackage{enumitem} % zur Formatierung von Aufzählungen
\usepackage{tikz} % Für Text in Kreisen, z.B. für Aufzählungen

%\usepackage[
%pict2e, %
%ngerman, %
%final, %draft|final
%]{struktex} % für Struktogramme (Algorithmenstrukturen)
%\AtBeginEnvironment{struktogramm}{\small} % Schrifteinstellungen für Struktogramme

%\AtBeginEnvironment{table}{\small}
%\AtBeginEnvironment{threeparttable}{\small}
%\AtBeginEnvironment{mtabular}{\small}

\usepackage{ifthen} % Für bedingte Anweisungen

% Für richtige Literaturverweise mit Links
%\usepackage{cite}
\usepackage[numbers,sort&compress]{natbib}
%\usepackage[
%  backend=biber,
%  bibstyle=trad-plain,
%  citestyle=numeric-comp ,
%  dashed=true,
%  sorting=none
%]{biblatex}
%\addbibresource{Literatur_Diss.bib}
\usepackage[pdftex,plainpages=false,pdfpagelabels=true,pdfnewwindow]{hyperref} % Hyperref so weit wie möglich hinten einbinden

% Für Koma-Zeitangaben (insb. für die Angabe von Datum in Zeit der Entwurfsversionen
%\usepackage{scrdate} % Funktion \today bereits in MikTex-Kernel enthalten
\usepackage{scrtime}

% Informationen für pdf-Dokument
\usepackage{hypcap}    % Damit in Bilder an die obere Kante gesprungen wird
\hypersetup{%
       pdfauthor={Willi Zschiebsch},
       pdftitle={Dissertation},
       pdfcreator={pdfTeX},
       pdfsubject={Titel der Dissertationsschrift}, % TODO: change title
       pdfkeywords={FKV, FVW, FVK, GFK, CFK, FEM, BEM, Bauteil, Auslegung, Messung}, % TODO: change keywords
       plainpages=false,
       bookmarksopen=true,
       bookmarksnumbered=true,
       bookmarksopenlevel=2,       
       plainpages=false,
			 pdfpagemode=FullScreen,
       pdfview=FitV,
       pdffitwindow=true,
       citecolor=blue,
       urlcolor=blue,
       pdfmenubar=true,
       pdfwindowui=true,
       pdfhighlight=/I,
       pdfborder={0 0 0},
       pdfpagemode=UseOutlines,
    	 pdfstartview=Fit,
			 linktocpage=true			
}
%\pdfminorversion=6 %Fehlermeldungen beim Einlesen von pdf-Bildern ausblenden, die mit pdf-Veriosn 1.5 und später erstellt wurden
\typeout{Anpassungen von mda, Stand 10.10.2012 (teilweise basierend auf TEXDEFS.TEX)}

\newboolean{Entwurfsmodus}
\setboolean{Entwurfsmodus}{true} % true|false

\definecolor{lightgray}{rgb}{0.5,0.5,0.5}
\definecolor{lightgrey}{rgb}{0.5,0.5,0.5}
\definecolor{darkred}{rgb}{0.6,0,0}

%\newcommand{\avg}[1]{\left< #1 \right>} % for average
\newcommand{\rot}[1]{\textcolor{red}{#1}} %rote Hervorhebung

% Breitenangabe fuer Strukturgramme
% 	Es ist leider nicht moeglich die Breite eines Strukturgramms auf die Textbreite zu beziehen, daher 150 fuer A4, und 105 fuer A5 einstellen
%		sProofOn muss trotzdem noch fuer jedes Diagramm wegen der Hoeheneinstellung durchgefuehrt werden!!!
\newcommand{\struktbreite}{150}

% Fuer eigene Verwendung
\ifthenelse{\boolean{Entwurfsmodus}}
{
\typeout{Entwurfsmodus an.}
\newcommand{\korr}[1]{\textbf{\textcolor{magenta}{#1}}} %rote Hervorhebung
\newcommand{\kommentar}[1]{ \textcolor{green}{\textit{(#1)}} } % Kommentar; in Endfassung einfach Inhalt der geschweiften Klammer loeschen
\newcommand{\fertig}{\texorpdfstring{ \textcolor[rgb]{0,0.8,0}{(\checkmark)}}{ o.k.}} % macht ein Haekchen innnherhalb von Ueberschriften
\newcommand{\inArbeit}{\texorpdfstring{ \textcolor[rgb]{0.8,0,0}{(\checkmark)}}{ i.A.}} % macht ein Haekchen innnherhalb von Ueberschriften
\newcommand{\fehlt}{\texorpdfstring{ \textcolor[rgb]{0.8,0,0}{(\textbf{!!})}}{ fehlt}} % macht ein Haekchen innnherhalb von Ueberschriften
\newcommand{\alt}[1]{\underline{ALT:} \textcolor{lightgray}{#1}} %alte Textpassagen
\newcommand{\neu}[1]{\underline{NEU:} \textcolor{darkred}{#1}} %neue Textpassagen
}{
%%%%% Fuer Ausdruck und Weitergabe
\typeout{Entwurfsmodus aus.}
\newcommand{\korr}[1]{} %rote Hervorhebung
\newcommand{\kommentar}[1]{} % Kommentar; in Endfassung einfach Inhalt der geschweiften Klammer loeschen
\newcommand{\fertig}{} % macht ein Haekchen innnherhalb von Ueberschriften
\newcommand{\inArbeit}{} % macht ein Haekchen innnherhalb von Ueberschriften
\newcommand{\fehlt}{} % macht ein Haekchen innnherhalb von Ueberschriften
\newcommand{\alt}[1]{} %alte Textpassagen
\newcommand{\neu}[1]{#1} %neue Textpassagen
}

%% Sonstige Definitionen
% Original war hinter Bild Tabelle Abschnitt immer ein ~ - also ein nicht trennbares Leerzeichen
\newcommand{\bild}[1]{Bild \ref{#1}} % Bildverweise
\newcommand{\kreis}[1]{\unitlength1ex\begin{picture}(2.5,2.5) \put(0.75,0.75){\circle{2.5}}\put(0.75,0.75){\makebox(0,0){#1}}\end{picture}} % Einekreiste Zahl oder Buchstabe
\newcommand{\tabelle}[1]{Tabelle \ref{#1}} % Tabellenverweise
\newcommand{\abschnitt}[1]{Abschnitt \ref{#1}} % Gliederungsverweise
\newcommand{\anhang}[1]{Anhang \ref{#1}} % Verweise zum Anhang
\newcommand{\name}[1]{\textsc{#1}} % Namentliche Nennungen
\newcommand{\fett}[1]{\textbf{#1}} % Fett
\newcommand{\ul}[1]{\underline{#1}} % unterstrichen --> in Paket soul bereits definiert
\newcommand{\dul}[1]{\underline{\underline{#1}}} % Doppelt unterstrichen
\newcommand{\mat}[1]{\pmb{#1}} % Kennzeichnung von Matrizen besser als \pmb (poor mans bold) waere \boldsymbol, welches allerdings leider nicht funktioniert -> fehlende Schriftart???
\newcommand{\kmplx}[1]{\ul{#1}} % Kennzeichnung komplexer Groessen
\newcommand{\kkmplx}[1]{\kmplx{#1}^\ast} % Kennzeichnung einer konjungiert komplexen Groesse
\newcommand{\oampl}[1]{\hat{#1}} % Ortsamplitude
\newcommand{\zampl}[1]{\check{#1}} % Zeitamplitude
\newcommand{\avrg}[1]{\overline{#1}} % Mittelwert
\newcommand{\gl}[1]{(\ref{#1})} % Gleichungsverweis
\newcommand{\UD}{\text{UD}}	% UD in Formeln (unidirektional)
\newcommand{\TV}{\text{TV}} % TV in Formeln (textilverstaerkt)
\newcommand{\EX}{\text{Exp.}} % Exp. in Formeln (experimentell ermittelt)
\newcommand{\FE}{\text{FE}} % FE in Formeln
\newcommand{\DS}{\text{DS}} % Deckschicht DS in Formeln
\newcommand{\KS}{\text{KS}} % Kernschicht DS in Formeln
\newcommand{\MAC}{\text{MAC}} % MAC in Formeln
\newcommand{\Terz}{\text{Terz}} % MAC in Formeln
\newcommand{\Wert}{\text{Wert}} % Formelzeichen Wert
\newcommand{\Geni}{{\text{Gen}_i}} % Gen_i in Formeln
\newcommand{\Kind}{\text{Kind}} % Kind in Formeln
\newcommand{\Elter}{\text{Elter}} % Elter in Formeln
\newcommand{\Lag}{\text{Lag}} % Lag fuer Lagerung in Formeln
\newcommand{\Mat}{\text{Mat}} % Mat fuer Material in Formeln
\newcommand{\DEVOP}{\name{DevOP}\xspace} % DEVOP- im Text

%-------------------------------------------------------------------
% Globale Trennvorschlaege
\hyphenation{Donau-dampf-schiff-fahrt}
\hyphenation{Ur-instinkt}
\hyphenation{ei-nen}
\hyphenation{vi-bro-akus-tisch}
\hyphenation{vi-bro-akus-tische}
\hyphenation{vi-bro-akus-tisch-en}
\hyphenation{vi-bro-akus-tisch-em}
\hyphenation{vi-bro-akus-tisch-es}
\hyphenation{vi-bro-akus-tisch-er}
\hyphenation{Vi-bro-akus-tik}
\hyphenation{Mehr-schicht-ver-bund}
\hyphenation{Straf-funk-tion-en}
\hyphenation{an-iso-trop}
\hyphenation{an-iso-tro-pe}
\hyphenation{an-iso-tro-pen}
\hyphenation{an-iso-tro-pem}
\hyphenation{an-iso-tro-per}
\hyphenation{An-iso-tro-pie}
\hyphenation{uni-di-rek-tio-nal}
\hyphenation{uni-di-rek-tio-nale}
\hyphenation{uni-di-rek-tio-nal-en}
\hyphenation{Leicht-bau-struk-tur-en}
\hyphenation{vis-ko-elas-tisch}
\hyphenation{be-an-spruch-ten}
\hyphenation{Stich-pro-ben-grö-ße}
\hyphenation{me-si-o-buk-kal}
\hyphenation{pa-la-tal}
\hyphenation{ste-re-o-mi-kros-ko-pi-sche}

%-------------------------------------------------------------------
%% Ligaturen - leider nicht global einstellbar
%% ff fi fl ffi ffl ll lll
% Schall"|leistung
% werkstoff"|inhärent
% Werkstoff"|integration


%-------------------------------------------------------------------
%\newcommand{\entspr}{\widehat{=}}
\newcommand{\entspr}{\stackrel{\scriptscriptstyle\wedge}{=}}
\newcommand*{\dunderl}[1]{\underline{\underline{#1\!}}\,}
\renewcommand{\Re}{\mbox{Re}} %normalerweise: Schn"orkel-R
\renewcommand{\Im}{\mbox{Im}} %normalerweise: Schn"orkel-I
\renewcommand{\i}{{\rm i}}    %normalerweise: i ohne Punkt
\newcommand{\de}{{\mathsf d}}  %sans serif für elektrische Groessen
\newcommand{\e}{{\mathsf e}}
\newcommand{\E}{{\mathsf E}}
\newcommand{\U}{{\mathsf U}}
\newcommand{\T}{{\rm T}} % Transponiert-Operator
\newcommand{\N}{\mbox{I\hspace{-0.19em}N}} % Menge N = nat"urliche Zahlen
\newcommand{\R}{\mbox{I\hspace{-0.19em}R}} % Menge R = reelle Zahlen
\newcommand{\C}{\mbox{\hspace{0.35em}\rule{0.04em}{1.55ex}\hspace{-0.35em}C}} % Menge C = komplexe Zahlen
\newcommand{\mitDelta}{\mathnormal{\Delta}}  % neu: LaTeX 2e
\newcommand{\mitTheta}{\mathnormal{\Theta}}
\newcommand{\mitPhi}{\mathnormal{\Phi}}
\newcommand{\mitPsi}{\mathnormal{\Psi}}
\newcommand{\brmitTheta}{\breve{\mitTheta}}
\newcommand{\brmitPhi}{\breve{\mitPhi}}
\newcommand{\brmitPsi}{\breve{\mitPsi}}
\newcommand{\brrho}{\breve{\rho}}
% fuer Buchstaben mit Kreisen
\newcommand*\mycirc[1]{%
\begin{tikzpicture}[baseline=(C.base)]
\node[draw,circle,inner sep=1pt,minimum size=3ex](C) {#1};
\end{tikzpicture}}

%\renewcommand{\theta}{\fehlertheta}      % \
%\renewcommand{\theta}{\vartheta}
\renewcommand{\phi}{\varphi}
%\renewcommand{\phi}{\fehlerphi}          %  zur Sicherheit
%\renewcommand{\epsilon}{\textepsilon}  % /
%\renewcommand{\epsilon}{\fehlerepsilon}  % /
% Das Symbol \phi (hier \varvarphi) wird fuer Sonderfaelle benoetigt.
% Folgende Definition entstammt fontmath.ltx.
\DeclareMathSymbol{\varvarphi}{\mathord}{letters}{"1E}
\DeclareMathSymbol{\eps}{\mathsf}{letters}{"0F} 
    
% Neue Spaltenarten für Tabellen
\newcolumntype{C}[1]{>{\centering\arraybackslash}p{#1}} % Mehrzeilige horizontal zentrierte Tabellenspalten
\newcolumntype{M}[1]{>{\centering\arraybackslash}m{#1}} % Mehrzeilige horizontal und vertikal zentrierte Tabellenspalten
%\newcolumntype{V}[1]{>{\RaggedRight\arraybackslash}m{#1}} % Mehrzeilige linksbündige und vertikal zentrierte Tabellenspalten
\newcolumntype{L}[1]{>{\RaggedRight\arraybackslash}p{#1}} % Mehrzeilig linksbündige oben ausgerichtete Spalten
\newcolumntype{R}[1]{>{\RaggedLeft\arraybackslash}p{#1}} % Mehrzeilig linksbündige oben ausgerichtete Spalten
\newcolumntype{d}{D{.}{,}{-1}}													% Ausrichtung am Dezimalpunkt und Satz als Komma
\newcommand{\tabkopf}[1]{\textbf{#1}}
\newcommand{\Dkopf}[1]{\multicolumn{1}{c}{\textbf{#1}}}% tabkopf für Spaltentyp D
\newcommand{\DkopfZU}[2]{\multicolumn{1}{M{#1}}{\textbf{#2}}} % tabkopf mit Zeilenumbruch
\newcommand{\DZU}[2]{\multicolumn{1}{M{#1}}{#2}} % tabkopf mit Zeilenumbruch
\newcommand{\Dtext}[1]{\multicolumn{1}{c}{#1}} 				% normaler Text im Spaltentyp D
\newcommand{\DSkopf}[2]{\multicolumn{1}{M{#1}|}{\textbf{#2}}}% tabkopf für Spaltentyp S mit Umbruchmöglichkeit
\newcommand{\DStext}[2]{\multicolumn{1}{M{#1}|}{#2}} 				% normaler Text im Spaltentyp S mit Umbruchmöglichkeit
% Einfacher Abstand in Tabellenumgebung mtabular
\newenvironment{mtabular}{%
	\small%
  \renewcommand*{\arraystretch}{1.4}%
  \tabular
}{%
  \endtabular
}

\pdfsuppresswarningpagegroup=1 %mk Unterdrueckung warning of multiple PDF files

\usepackage{lscape} %mk Tabelle Querformat
\usepackage{amsmath} %mk for boldsymbol

\usepackage[percent]{overpic} %mk Bild auf ganzer Seite

\ifthenelse{\boolean{Entwurfsmodus}}{
	\hypersetup{%
			colorlinks=true, % true|false
      linkcolor=blue,	% blue|black       
      citecolor=blue,		% blue|black
	}
} {
	\hypersetup{%
			colorlinks=false, % true|false
      linkcolor=black,	% blue|black       
      citecolor=black,		% blue|black
	}
}

%**************************************************************************
%**************************************************************************
%*********** BEGINN des Dokumenteninhaltes ********************************
%**************************************************************************
%**************************************************************************
\begin{document}

\pagenumbering{gobble} % Ausschalten der Seitenzählung für die Titelseite und das Vorwort

\begin{titlepage}

\centering
\vspace*{-2.0cm}
%\vfill

{\renewcommand{\baselinestretch}{1.4}\LARGE
{\textbf{Titel}}\par
}

\vspace{1cm}

Von der Fakultät Maschinenwesen \\[+0.5\baselineskip]
der\\[+0.5\baselineskip]
Technischen Universität Dresden \\[+0.5\baselineskip]
zur\\[+0.5\baselineskip]
Erlangung des akademischen Grades \\[+0.5\baselineskip]
Doktoringenieur (Dr.-Ing.) \\[+0.5\baselineskip]
angenommene Dissertation

\vspace{1cm}

von \\[+0.5\baselineskip]
{\large M. Eng.\ Willi Zschiebsch} \\[+0.5\baselineskip]
geboren am 28. Oktober 1996 in Wurzen \\

\vspace{1cm}

\begin{tabbing}
\hspace*{4.3cm}\= \kill
Tag der Einreichung: \> TT.MM.JJJJ \\
Tag der Verteidigung: \> TT.MM.JJJJ 
\end{tabbing}

\vspace{1cm}

\raggedright
Promotionskommission:
\begin{tabbing}
\hspace*{2.8cm}\= \kill
Vorsitzender: 	\> Prof. habil. Dr.-Ing. Maik Gude\dots \\[+0.5\baselineskip]
Gutachter:    	\> Prof. Dr.-Ing. Niels Modler \dots \\
				\> Prof. habil. Dr.-Ing. Robert Böhm \\[+0.5\baselineskip]
Beisitzer: 		\> Prof. \dots \\
							\> Prof. \dots \\
\end{tabbing}
\end{titlepage}


%--------------------------------------------------------------------------
\hypersetup{pageanchor=false}
\thispagestyle{empty}
\cleardoublepage

\thispagestyle{empty}

\vspace*{\fill}\vspace*{\fill}\vspace*{\fill}\vspace*{\fill}\vspace*{\fill}

\subsection*{Vorwort}

Die vorliegende Arbeit entstand während meiner Zeit als Promotionsstipendiat in der Arbeitsgruppe Leichtbau mit Verbundstoffen unter der Leitung von Prof. Dr.-Ing. habil. Robert Böhm. Finanziert wurde die Arbeit von der HTWK im Rahmen des HTWK-Promotionsstipendiums.

Meinem hochverehrten Doktorvater, Herrn Prof. Dr.-Ing. Niels Modler vom Institut für Leichtbau und Kunststofftechnik der Technischen Universität Dresden, möchte ich meinen besonderen Dank für die kontinuierliche Unterstützung ausdrücken.

Hervorzuheben ist die Unterstützung meiner Kollegen im ElViS-Projekt, insbesondere Johannes Kühn, Robert Seidel-Greiff, Daniel Wolz und Thomas Behnisch, für die vielen fachlichen Diskussionen, Hinweise und ihren großen Einsatz, der in die experimentelle Validierung der in dieser Arbeit entwickelten Methode geflossen ist.

Großer Dank gebührt auch meinen vielen Kollegen, die mir während meiner Forschungsarbeit motivierend zur Seite standen. Besonders hervorzuheben sind die Beiträge von Philipp Johst, Davood Peyrow Heyadati, Saskia Roßberg und Peter Jakob.

Abschließend bedanke ich mich sehr herzlich für den bedingungslosen Beistand und die Nachsicht meiner Freunde und meiner Familie, die für mich einen wichtigen Rückhalt und Ausgleich bedeuteten.

\cleardoublepage
\hypersetup{pageanchor=true}
\hypersetup{pageanchor=false}
\chapter*{Kurzfassung}
\thispagestyle{empty}

Das

\textbf{Schlagworte:} 

\vspace{1cm}
{\bfseries \LARGE Abstract}
\vspace{1cm} 

The

\textbf{Keywords:} 

\cleardoublepage
\hypersetup{pageanchor=true}

\pagenumbering{roman} % Römische Zahlen für den ganzen Vorspann

\pdfbookmark[1]{\contentsname}{toc}	% Link zum Inhaltsverzeichnis in der pdf-Datei
\tableofcontents

%\clearpage
%\listoffigures % Abbildungsverzeichnis
%
%\clearpage
%\listoftables % Tabellenverzeichnis

\clearpage
%\chapter*{Symbolverzeichnis}
%\label{sec:Symbolverzeichnis}
\addchap{Symbolverzeichnis}
\markboth{Symbolverzeichnis}{Symbolverzeichnis}


%Einzug erste Spalte
\newlength{\TabulatorVZ} % definiert einen neuen L�ngenparameter
\settowidth{\TabulatorVZ}{$m$, $n$, $i$, $\imath$, ${\bar{\imath}}$, $\jmath$\quad} % Setzt den L�ngenparameter auf den Wert, den der Text hat
% Einzug Einheitenspalte
\newlength{\TabulatorEH} % definiert einen neuen L�ngenparameter
\setlength{\TabulatorEH}{\widthof{$\si{\celsius}$; $\si{\kelvin}$}}

\newlength{\TabulatorTX}
\setlength{\TabulatorTX}{\textwidth}
\addtolength{\TabulatorTX}{-\TabulatorVZ-\TabulatorEH-2\tabcolsep}

{\renewcommand*{\arraystretch}{1.2}%

\section*{Abkürzungen}

\begin{longtable}{@{}p{\TabulatorVZ}@{}p{\TabulatorTX+\TabulatorEH+2\tabcolsep}@{}}
FE								& Finite Elemente \\
FEM								& Finite"=Elemente"=Methode \\
PA/PA6						& Polyamid/Polyamid-6 \\
PEEK    	        & Polyetheretherketon \\
PP								& Polypropylen

\end{longtable}

\section*{Allgemeine Notation}

\begin{longtable}{@{}p{\TabulatorVZ}@{}p{\TabulatorTX+\TabulatorEH+2\tabcolsep}@{}}
a									& Skalar \\
\textbf{a}				& Tensor 1. Stufe (Vektor)
\end{longtable}

\section*{Lateinische Buchstaben}

\begin{longtable}{@{}p{\TabulatorVZ}@{}p{\TabulatorTX}p{\TabulatorEH}@{}}
	$A$					& Fläche					& $\si{\metre\squared}$ \\
	$C$					& Capazität					& $\si{\ampere\s}$					\\
	$c$					& Konzentration				& $\si{\mole\per\metre\cubed}$\\
	$D$					& Diffusionskonstante		& $\si{\metre\squared\per\second}$\\
	$E$					& Elastizitätsmodul			& $\si{\pascal}$	\\
	$F$					& Kraft						& $\si{\newton}$	\\
	$F_{\text{K}}$		& Faraday-Konstante			& $\si{\coulomb\per\mole}$	\\
	$f_{\pm}$			& Aktivitätskoeffizient		& $\si{\coulomb\per\mole}$	\\
	$G$					& Schubmodul				& $\si{\pascal}$	\\
	$h$					& Plattendicke				& $\si{\metre}$		\\
	$\textbf{i}$		& Stromdichte				& $\si{\ampere\per\metre\squared}$		\\
	$j$					& molare Ionenflussdichte	& $\si{\mole\per\metre\squared\per\second}$		\\
	$K$					& Konstante					& -					\\
	$R_{\text{K}}$		& Unverselle-Gaskonstante	& $\si{\joule\per\mole\per\kelvin}$	\\
	$U_{\theta}$ 		& Elektrochemisches Standardpotenzial	& $\si{\volt}$ \\
	$V$					& Volumen					& $\si{\cubic\metre}$
\end{longtable}

\section*{Griechische Buchstaben}
\begin{longtable}{@{}p{\TabulatorVZ}@{}p{\TabulatorTX}p{\TabulatorEH}@{}}
	$\varepsilon$			& Dehnung						& -	\\
	$\vartheta$				& Temperatur					& $\si{\celsius}$; $\si{\kelvin}$ \\
	$\sigma$				& elektrische Leitfähigkeit		& $\si{\pascal}$ \\
	$\boldsymbol{\sigma}$	& Mechanischer Spannungstensor	& $\si{\pascal}$ \\
	$\sigma_{\text{B,K}}$	& Boltzmann"=Konstante 			& $\si{\joule\per\kelvin}$ \\
	$\nu$					& Poissonzahl					& -	\\
	$\rho$					& Dichte						& $\si{\kilo\per\metre\cubed}$
\end{longtable}

\section*{Indizes, Exponenten und mathematische Akzente}

\begin{longtable}{@{}p{\TabulatorVZ}@{}p{\TabulatorTX+\TabulatorEH+2\tabcolsep}@{}}
	$x_{\text{AM}}$				& Aktivmaterial\\
	$x_{\text{b}}$				& Binder\\
	$x_{\text{DS}}$				& Deckschicht\\
	$x_{\text{echem}}$			& elektro"=chemisch\\
	$x_{\text{exp}}$			& experimentell\\
	$x_{\text{l}}$				& leitende Phase\\
	$x_{\text{s}}$				& specihernde Phase\\
	$x_{\text{mech}}$			& mechanisch\\
	$x_n$						& Normal zur Oberfläche	\\
	$x_{-}$						& negative Elektrode \\
	$x_{\text{OCV}}$			& Gleichgewichtsspannung (\textit{engl.} open-circuit voltage) \\
	$x^{+}$						& positive Elektrode \\
	$x_{\text{th}}$				& thermisch \\
	$x_{\text{theor}}$			& theoretisch \\
	$\vec{x}^T$					& Transponierter Vektor	\\
	$\tilde{x}$				    & Effektivwert 	\\
	$\avrg{x}$					& Mittelwert
\end{longtable}

} % Ende arraystrecth


\clearpage

\cleardoublepage % sonst erhält das Tabellenverzeichnis schon die erste arabische Seitenzahl...
\pagenumbering{arabic} % ab hier Arabische Zahlen für den Hauptteil

%***************************************************************************************************
%********************  EINLEITUNG  *****************************************************************
%***************************************************************************************************
\chapter[Einleitung]{\label{sec:Einleitung}Einleitung}
% max 1 Seite

Die 

%***************************************************************************************************
%********************  Motivation und Zielstellung  ************************************************
%***************************************************************************************************
\section{\label{sec:Motivation_Zielstellung}Problemstellung und Zielsetzung}
% rund 1,5 Seiten inklusive Bild

Die 

%***************************************************************************************************
%********************  LITERATUR�BERSICHT  *********************************************************
%***************************************************************************************************
\section{\label{sec:Stand_Forschung_Technik}Literaturübersicht}
% 3-5 Seiten

Die









   


\chapter{Stand der Forschung}

Im folgenden Kapitel wird ein grundlegendes Verständnis für die Funktionsweise von Strukturbatterien vermittelt werden. Außerdem werden die Besonderheiten gegenüber konventionellen Batterien oder faserverstärkten Verbundwerkstoffen erläutert. Dazu werden die wichtigsten Eigenschaften und ihre Ermittlungsverfahren erläutert und Rolle der Einzelkomponenten im Zusammenhang der Materialauswahl näher erklärt. Anschließend werden aktuelle Entwicklungsansätze diskutiert und abschließend die ungelösten Herausforderungen mit den aktuellen Methoden näher analysiert.

\section{Grundlagen der Strukturbatterie}
Strukturbatterien sind 

\section{Wichtigsten Eigenschaften und ihre Ermittlungsverfahren}
%\subsection{Interkalation}
\subsection{Gesamtkapazität}
\subsection{Energiedichte}
\subsection{Elektrische Spannung}
\subsection{Zyklenverhalten}
\subsection{Steifigkeit und Festigkeit}
%\subsection{Mechanische Spannung}
\subsection{Multifunktionale Effizienz}

\section{Materialauswahl}

Jedes  Material in einer Strukturbatterie erfüllt mehrere Aufgaben gleichzeitig. Die am meisten benutzte Untergliederung teilt die Materialien nach ihrer elektrochemischen Rolle ein.


\subsection{Anode}
Die Anode sollte ein niedriges elektrochemisches Potenzial und eine schnelle Interkalation für eine möglichst hohe Energiedicht und Leistungsdichte aufweisen. Zusätzlich profitieren Strukturbatterien sehr von Anoden mit hohen Festigkeits- und Steifigkeitswerten.

Die Verwendung von Kohlenstoff in Lithium-Ionen Batterien wurde erstmalig von \textsc{Yoshino} \cite{Yoshino1986} 1886 veröffentlicht der für diesen Durchbuch 2019 den Nobelpreis erhielt.
Heute ist Kohlenstoff eines der meist benutzen Material in wiederaufladbaren nicht-aqueous Batterien \cite{Ahmad2021}. Am meist verbreitesten ist dabei die Kombination von Grafit als Anode und einer Kathode aus Phosphat, welche eine maximale Energiedichte von 200-250~$\si{\watt \hour \per \kg}$. 
Es gibt zwei Arten von Kohlenstoff die in der Lage sind Ionen einzulagern: geordenten und ungeordenten \cite{Ghosh2024}.

Geordenter Kohlenstoff sind Materialien mit einer weitreichenden Ordnung ud hoger Kristallinität. Die Ordnung kann sich dabei auf eine Achse (CNTs), eine Ebene (Graphen) oder den Raum (Grafit) begrenzen \cite{Wang2021}.

Grafit hat eine hoch-kristaline Struktur und besitzt eine weit reichende Ordnung. Die $\text{sp}^\text{2}$-hybritisierten Graphenelagen sind in lang der c-Achse gestappelt und folgen entweder die hexagona AB-Sequenz oder dei rhombohedrale ABC-Folge. Die binden $\pi$-Orbitale ermögliche eine gute Leitfähigkeit von $10^3$-$10^4$~$\si{\siemens \per \cm}$ in Ebenenrichtung. Die Graphenebenen ahben einen Abstand von 3,35~$\si{\angstrom}$ entlang der c-Achse und werden nur durch relativ schwache van der Waals Kräfte (16-17~$\si{\kJ \per \mol}$) zusammen gehalten. Der relative hohe Abstand und die schwachen Bindungskräfte machen es einfach, damit kleine Atome wie Lithium oder Kalium sich zwischen den Ebenen einlagern können \cite{Wang2021}.

Der Interkalationsprozess läuft dabei in vier Stufen ab, was sich im Potenzialverlauf erkennen lässt. Das Lithium-Ion wird dabei zwischen zwei benachbarten Graphene-Ebenen eingelager, wobei jedes Lithium-Ion den niedrigsten Energiezustand einnimmt, der im Zentrum eines hexagonalen Kohlenstoffring existiert \cite{Sole2014,Weng2023}. Allerdings können, Lithium-Ionen sich nicht durch die Grapheneschichten hindurchtunneln, wehalb sie die Transportbewegung zwischen die Schichten nur entlang von Gitterdefekten möglich ist \cite{Nishidate2005}. Die Einlagerungsgeschwindigkeit ist dabei nicht konstant und kann während jeder Stufe um teilweise das tausendfache einbrechen \cite{Levi1997}. Dieses Verhalten kommt nach \textsc{Aurbach et al.} durch die Bildung von Lithiumklustern zwischen den beiden Graphenschichten zu stande, welche die Diffusion weiterer Lithiumionen am Anfang einer neuen Phase verhindern \cite{Markevich2005}.  Das maximale Einlagerungsmenge ist mit der $\text{LiC}_\text{6}$-Konfugration erreicht, bei der zwischen jeder Grafitschicht alle möglichen Plätze beleget wurden. Die Menge an eingelagerten Lithiumion entspricht dabei  einer theoretischen spezischen Kapazität von 372~$\si{\mA \hour \per \g}$ \cite{Winter1998}. 
Eine weitere wichtige Eigenschaft ist die relative hohe Dichte von >2~$\si{\g \per \cm \cubed}$, was dabei hilft möglichst viel Aktivmaterial in kleinem Raum zu haben um kleine Batterien mit einer hohen Energiedichte zu erzeugen.

Seit seiner Entdeckung in 2004 \cite{Novoselov2004} ist Graphene zunehmend in den Fokus der Batterieforschung geraten mit einer theoretischen Kapazität von >1000~$\si{\mA \hour \per \g}$ hoher mechansicher Zugfestigkeit von $\approx$130~$\si{\GPa}$ und einer Zugsteifigkeit von $\approx$1~$\si{\tera \Pa}$ stellt es ein ideales Material für den Einsatz in Strukturbatterien da \cite{Novoselov2012}. Jedoch konnte das Material bisher nur im Labormaßstab und nur in unzureichenden Mengen syntehtisiert werden. Auch ist bisher umstritten, wie die Einlagerung von Lithium bei Graphene genau abläuft, was je nachdem die theoretische Kapazität noch stark noch oben oder unten korrigiert. Bisherige Experimente mit zweilagigen Graphene kommen zu unterschiedlichen Ergebnissen. \textsc{Ji et al.} beobachtete einen Mechanismus der auf einen ähnlichen Prozess wie bei Grafit vermuten lässt, während \textsc{Kühne et al.} sugenante super-dichte Lithiumeinlagerung zwischen den beiden Grapheneschichten gemessen haben will. Derzeitig geht die Produktion von Graphene nicht über den Labormaßstab hinaus und bleibt daher für den Einsatz in Strukturbatterien bis auf weiteres ungeeignet.

CNTs sind geordente 1D Kohlenstoffstrukturen, welche 1991 von \textsc{Iijima} \cite{Iijima1991} erstmalig entdeckt wurden. Diese zylindrischen Formen des Kohlenstoffes haben einen Durchmesser von 1-20~$\si{\nano\metre}$  und meist ein hohes Längen-zu-Durchmesser-Verhältnis, mit der höchsten bisher dokumentierten Länge von 55~$\si{\centi\metre}$ von \textsc{Zhang et al.} \cite{Zhang2013}. CNTs werden meist durch ihrer Schichtanzahl in SWCNT und MWCNT unterschieden. Darüber hinaus können SWCNTs je nach Winkel des graphenähnlichen Gitters im Mantels gegenüber der Zylinderachse metalische oder Halbleiterähnliche Eigenschaften aufweisen. 
SWCNTs und MWCNTs besitzen hohe spezifische Oberflächen (1300~$m^2/g$), eine sehr hohe elektische Leitfähigkeit (5000~$\si{\siemens \per \cm}$) und eine hohe Ionenleitfähigkeit von (>100000~$\si{\cm \squared \per \V \per \s}$) \cite{Xu2011,Uetani2014,Charlier2007}.

Ungeordneter Kohlenstoff hat keine weitreichende periodischen Strukturen in Ebenen oder Stapelrichtung. Sie bestehen hauptsächlich aus zufällig ausgerichten sp2 grafitischen Mikrobereichen und Verknüpfungen durch sp3 hybridiserte Kohlenstoffatome in amorphen Gebieten. Der Anteil der sp3-Verknüpfungen bestimmt ob eine Grafitisierung bei Temperaturen bis zu 3000~$\si{\degreeCelsius}$ möglich ist. Dies führt zu einer Unterscheidung in sogenannten harten oder weichen Kohlenstoff. 

Bei weichem oder grafitisierenden Kohlenstoff kann aufgrund der geringen Anzahl von sp3 Verknüfungen immer noch eine thermisch bedingten Mobilität der Kohlenstoffschichten erfolgen, was bei einer Wärmebehandlung von 1500-3000~$\si{\degreeCelsius}$ unter Sauerstoffausschluss(Pyrolyse) zu einer Umwandlung zu Grafit führt. Ein weitverbreiter Ansatz zu Herstellung von weichem Kohlenstoff ist die thermische Zersetzung von verschiednen organsichen Precursorn in einer inerten Atmosphare bei hohen Temperaturen (1000-1700 $\si{\degreeCelsius}$) (Karbonisierung). Besonders eigenen sich hierbei pyrolytische aromatische Verbindungen, wie etwa Pech, Benzol, Petrolkoks, Polyvinylacetat und Polyvinylchlorid \cite{Wang2021}. Die Wahl des sogenannten Precursormaterials und Prozessparameter haben maßgeblichen Einfluss auf chemische Strukut, die wiederum die Eigenschaften von ungeordenter Kohlenstoff bestimmt. Besonders entscheident ist hierbei der Kristallinitätsgrad oder Grafitisierungsgrad, welcher u.a. durch Ramanspektroskopie bestimmt werden kann \cite{Yu2014}. Die micro-kristalinen Grafitberiche haben dabei ein ein ähnliches Einlagerungsverhalten als Grafit. Die kleinere Menge an grafitischen Strukturen sorgt jedoch das die Ionenspeicherkapazität bei einer langsamen Beladung (C/10) von grafitischen Kohlenstoff nur etwa 250~$\si{\mA \hour \per \g}$ (Grafit 372~$\si{\mA \hour \per \g}$) erreicht. Jedoch ist die Einlagerung deutlich schneller, was bei höheren Beladungs- und Entladungstests (10C) zu einer drei Mal höheren Kapazität (weicher Kohelenstoff 90$\si{\mA \hour \per \g}$ und Grafit 25 $\si{\mA \hour \per \g}$) führt \cite{Schroeder2014}. Auch zeigt grafitisierender kohlenstoff im Gegensatz zu Grafit keine Einbrüche im Diffionsverhalten was für dafür spricht, dass die Einlagerung stufenlos erfolgt. Allerdings bleiben auch in den weniger geordneten Strukturen  mehr Lithiumionen gefangen, weshalb die CE während des ersten Zykluses für weichen Kohlenstoff nur bei etwa 72~\% (Grafit 82~\%) liegt. Jedoch ist nachdem Prelithierungspozess die CE auch hier bei über 99~\% \cite{Schroeder2014}. 

Harter oder nicht-grafitisierender Kohlenstoff lässt sich selbst bei hohen Karbonisierungstemperaturen (<3000~$\si{\degreeCelsius}$) nicht in Grafit umwandeln. Meist wird dieser aus der Karbonisierung von Precursoren mit wenigen aromatischen Strukturen, wie etwa Zuker, Holzkohle, Cellulose und Kokosnussschalen gewonnen \cite{Wang2021}. Die komplexeren Organischen Strukturen der Precursors sorgen, dass nach der Kabronisierung eine signifikante Anzahl an kleineren Poren und Risse in der Mikrostruktur verbleiben, die einen schnellen Zugang zu den Interkaltationsbereichen erlauben und für eine hohe aktive Oberfläche sorgen \cite{Liu2019a}. Graphenschichten $\approx$0.4~$\si{\nm}$ die ungeordente Mikrostuktur 
CR von 85~\% nach hundertausend Zyklen \cite{Cao2014}.

% Eine CAG ist ein hartcarbon?

%Eines der am frühsten und immer noch am weitverbreitesten Aktivematerialien anodenseitig ist Grafit. Zwischen den Grafitschichten können Lithiumionen eingelagert werden. In herkömmlichen monofunktionalen Batterien werden oft dünne Kupferfolien mit einer Grafitpartikelbeschichtung verwendet. Die zusätzliche Additive in der Pulvermischung halten die Partikel zusammen und sorgen für einen geringen Widerstand beim Transport der Elektronen zur Kupferelektrode. Die Bindungen zwischen den Partikeln sind jedoch sehr schwach und tragen nicht zur Steigerung der mechanischen Eigenschaften bei \cite{Chen2024}. Außerdem ~mAh/gsorgt die Ausdehnung infolge von Lithierung mit der Zeit für Risse durch die mit der Zeit der Leitungswiderstand steigt, was einer von vielen beobachten Alterungsmechanismen von Batterien ist \cite{Xiong2020}.

%Die begrenzte Kapazität, langsame Diffusionskinetik, geringe mechanische Eigenschaften, sind einige der Faktoren die Untersuchungen Kohlenstoff-Nanostrukturen und andere Morphologien bewegen.

Kohlenstofffasern sind einer der vielversprechesten Kandidaten für lastentragenden Anoden. Ca. 96~\% aller Fasern weltweit werden aus Polyacrylnitril (PAN) hergestellt, die Restlichen werden aus Precursorn wie Pech, Rayon oder Lignin gewonnen \cite{Das2016}. Kohlenstofffasern besitzen im Allgemeinen hohe Festigkeits- und Steifigkeitswerten, eine elektrische gut leitende Oberfläche, die mit 0,2~$\si{\metre \squared \per \g}$ zwar zu klein für Batterieanwendungen ist, jedoch durch verschiedene Oberflächenmodifikationen \cite{Qian2013,Senokos2023} auf über 200~$\si{\metre \squared \per \g}$ gesteigert werden kann \cite{Zenkert2024}. Jedoch haben Wahl des sogenanten Precursormaterials, sowie die Verfahrensparameter während des Spinnens, Stabilisierens und Karbonisierens einen entscheidenten Einfluss auf die Struktur der Faser, was sich wiederum signifikant in den mechanischen, elektrischen und elektrochemischen Eigenschaften bemerkbar macht \cite{Newcomb2015}.
Verallgemeinert lässt sich feststellen, dass ein höherer Anteil an kristallien Grafitstrukturen in der Faser zu einer höheren Steifigkeit, Festigkeit, sowie thermische und elektrischen Leitfähigkeit führt. Jedoch ist die Kapazität von 150~$\si{\mA \per \g}$ (C/10) bei diesen hochmoduligen Faser, wie etwa M60J, deutlich geringer, als bei fasern mit niedrigem Kristalinitätsanteil, wie etwa T800 (265~$\si{\mA \per \g}$) und IMS65 (358~$\si{\mA \per \g}$) \cite{Fredi2018}. Man nimmt an, dass die geringer Kapazät durch die sich wie ein Mantel um die Faserausbildenten relativ großen Kristallstrukturen und turbostatischen Graphitstrukturen zu stande kommt, die einem radialen Ionentransport stark behindern \cite{Zenkert2024}. Bei Fasern mit weniger starker Graphitkristallausbildung bieten die zahlreichen Gitterdefekt, ähnlich wie bei ungeordnetem Kohlenstoff, genug Zugang für die Lithiumionen um sich bei kleineren Beladungsraten vollstädnig einlagern zu können \cite{Fredi2018}. Dies deckt sich mit Beobachtungen, dass sich Lithium erst in den ungeordenrteren (amorphen) Bereichen einlagert und erst mit höhere beladung auch die grafitischen Strukturen besetzt werden \cite{Fang2022}. Wie auch bei grafitischen Kohelsntoffen verlieren Kohlenstofffaser einen großen Teil ihrer Ladungsträger während des ersten Zyklus \cite{Jacques2013}. Jedoch bleibt CE auch nach zehn Zyklen bei über 99,9~\% \cite{Hagberg2016}, was bedeutet, dass der weitere Beladungsprozess und Entladungsprozess nahezu verlustfrei ist. Allerdings hat die Einlagerung von Ionen auch zur Folge, dass sich die mechansichen Eigenschaften der Fasern ändern. Dabei verdoppelt sich E-Modul quer zur Faserrichtung im lithierten Zustand und ging nahezu vollständig auf die Werte im ursprungszustand während der delithierung zurück. Für das Modul in Faserrichtung konnten dabei allerdings keine Veränderunge gemessen werden \cite{Duan2021}. Weiterführende Zugversuchen im lithierten und delithierten Zustand zeigten außerdem, dass die Zugfestigeit während der Lithierung um 25-30~\% zurück ging, die selbst noch nach der Entladung um 5-10~\% geringer waren als im ursprünglichen Zustand \cite{Jacques2012}. Versuche mit verschiedenen Lithierungsgraden konnten dabei eine direkte Abhängkeit zur Zugfestigkeit festellen \cite{Jacques2014}, was darauf schließen lässt, das die durch die Einlagerung verursachten Dehnungen im Material maßgeblich den Festikeitsverlust beeinflussen \cite{Zenkert2024}. Der Festigkeitverlust im Zusammenhang einer multifunktionalen Nutzung muss damit zwar unbedingt berücksichtigt werden und ist spielt besonders bei steifigkeitsgetriebenen Anwenugen eine untergeordnete Rolle und eine weitere Degradierung der Fasern nicht beobachtet wurde \cite{Zenkert2024}.


\begin{table}[ht]
    \centering
    \caption{Übersicht bisher entwickelter Strukturbatterien.}
    \begin{tabular}[t]{lccc}
    \toprule
    &Spezifische Kapazität [mAh/g]&Spezifische Kapazität [mAh/g]&Referenz\\
    \midrule
    T300&170&91&\cite{Kjell2011}\\
    T300 unbeschichtet&350&130&\cite{Kjell2011}\\
    T800&170&98&\cite{Kjell2011}\\
    T800 unbeschichtet&194&112&\cite{Kjell2011}\\
    IMS65 &166&108&\cite{Kjell2011}\\
    IMS65 unbeschichtet&360&177&\cite{Kjell2011}\\
    \bottomrule
    \end{tabular}
\end{table}%

Conversion/alloying metalle wie etwa Silicium erreichen zwar Energiedichten >250Wh/kg jedoch ist ihre Aufnahme von Lithium mit großen Volumenänderungen verbunden, welche die Zyklenzahl drastisch reduziert.

\subsection{Kathode}

On the other hand, phosphate-based intercalation cathodes (LiFePO4 and LiMn1-xFexPO4) are the safest choice for high-power batteries. The robust phosphate framework undergoes minimal volume changes during de/lithiation, offers faster ionic diffusion, and does not release oxygen when damaged \cite{Ling2021}. However the material fails drastically in the absence of a conductive coating due to its poor electronic conductivity ($10^{\text{-}9}-10^{\text{-}11}$ S cm-1).

\subsection{Elektrolyte}
ionenleitfähigkeit und Festigkeit
Elektrolyteinterface
Sicherheitsaspekt bei zweiphasigen Elektrolyten (neg: Brennbarkeit Ionicliquid, pos: durch Wärme schmilzt bei Thermoplasitischen Systemen und verklept die Poren, was einen weiteren Ladungsaustausch bei Kurzschluss verhindern kann)
\subsection{Separator}

\begin{table}[h!]
    \caption{Properties of different types of separators}
    \label{tab_separator_comp}
    %\begin{adjustwidth}{-\extralength}{0cm}
    \newcolumntype{C}[1]{>{\hsize=#1\hsize\centering\arraybackslash}X}%
    \begin{tabularx}{\textwidth}{
    %C{0.6}
    C{1} 
    C{1.8} 
    C{0.8} 
    C{0.8} 
    C{0.8} 
    C{0.6}
    }
        \toprule
        \textbf{Type of separator}
        &\textbf{Separator material} 
        &\textbf{Ionic conductivity\textsuperscript{*} (mS/cm)} 
        &\textbf{Young's modulus\textsuperscript{*} (GPa)}
        & \textbf{Strength\textsuperscript{*} (MPa)}
        &\textbf{Ref.} \\
        \midrule
        %\legendsep{c0}&
        Glass fibre&Glass fibre&1.13&21
        &325
        &\cite{Deka2017}\\
        %\midrule
        \addlinespace
        %\legendsep{c10}&
        Polymer&RF/PLA&110&0.3271
        &15.2
        &\cite{Vargun2020}\\
        %\midrule
        \addlinespace
        %Gel polymer electrolyte&$\mathrm{PVA/KOH/K_3[Fe(CN)_6]}$&45.56&n.a.&n.a.&\cite{maHighPerformanceSolidstate2014}\\
        %%\midrule
        %\legendsep{c4}&
        Solid polymer electrolyte&$\mathrm{PEGDGE/TETA/EMIBF_4}$&0.2&26
        &350
        &\cite{Hubert2022, Choi2022}\\
        %\midrule
        \addlinespace
        %\multirowcell{2}{\legendsep{c6}}&
        \multirowcell{2}{Ceramic}
            &$\mathrm{PVDF/PPG/LiCl/CaTiO_3}$&n.a.&1.2
            &65
            &\cite{Alvarez‐Sanchez2019}\\
            &$\mathrm{PVB/Al_2O_3NW}$&13.5&n.a.
            &30
            &\cite{Liu2020a}\\
        %\midrule
        \addlinespace
        %Diode-like polymer electrolyte&PVP/PEI/SWCNT&n.a.&n.a.&n.a.&\cite{chowdhurySupercapacitorsElectricalGates2019}\\
        %%\midrule
        %Ceramic&NPs/PTFE/SiC&n.a.&n.a.&1.3&\cite{qinCeramicBasedSeparatorHighTemperature2018,zhaoInorganicCeramicFiber2017}\\
        %%\midrule
        %Tree-leave&Quercus rubra&n.a.&n.a.&n.a.&\cite{chenTrashTreasureFallen2022,wangMechanicalCharacteristicsTypical2010}\\
        %%\midrule
        %Eggshell membrane&Eggshell membrane&3.8&n.a.&6.59&\cite{yuUsingEggshellMembrane2012}\\
        %%\midrule
        %\legendsep{c8}&
        Cellulose&MCC/AMIM-Cl&298.6&5.43
        &71.71
        &\cite{Ahankari2022, Xu2020}\\
        %%\midrule
        %Graphene oxide&Graphene oxide paper&n.a.&n.a.&n.a.&\cite{shulgaSupercapacitorsGrapheneOxide2015,comptonTuningMechanicalProperties2012}\\
        %%\midrule
        %Metal-organic framework&Metal-organic framework&n.a.&n.a.&n.a.&\cite{mengMetalOrganicFrameworks2015,bundschuhMechanicalPropertiesMetalorganic2012}\\
        \bottomrule
    \end{tabularx}
    %\end{adjustwidth}
    \noindent{\footnotesize{\textsuperscript{*} The abbreviation not available (n.a.) is used.}}
\end{table}

\subsection{Pouchfolie}
Herkömmliche Pouchzellen sind mit einer kunststoffbeschichteten Aluminiumhülle vor Umwelteinflüssen geschützt. Insbesondere verhindert diese das Feuchtigkeit in die Batterie eindringt und giftige oder brennbare Stoffe aus der Batterie entweichen können. Außerdem ermöglichen die guten mechanischen und Wärmeleiteigenschaften der Alumiumfolie eine geringe Gesamtmasse und eine effizientere Temperaturregulierung der Zellen. Eine zunehmend wichtiger werdende Aufgabe, die allerdings noch nicht hinreichend erfüllt, wird ist das Aufbirngen einen äußeren Zelldruckes.
In mehrere Studien konnte gezeigt werden, dass durch einen hohen externen Druck die Kontaktierung zwischen Elektrode und Elektrolyte verbessert wird, was einen besseren Ionen- und Elektronentransport bewirkt. Außerdem können ungewünschte Nebenreaktionen unterdrückt werden, wie etwa Gasbildung und Dendritwachstum, was den Lithiumverlust beim Laden und Entladen reduziert und somit dem Kapazitätsverlust entgegenwirkt und das Batterieleben verlängert \cite{Mussa2018,Mueller2019,Sakamoto2019}.
Besonders Batterien mit Feststoffelektrolyten benötigen einen deutlich höherer Druck um den Kontakt zwischen Elektrode und Elektrolyte zu gewährleisten \cite{Boaretto2021}. Jedoch existiert zurzeit noch keine zufriedenstellende Lösung. Zwar wird bereits bei der Herstellung mittels verpressen der Elektroden ein gewisser Druck realisiert, allerdings können größere Drücke damit nicht appliziert werden oder über längere Zeit aufrechterhalten werden \cite{Garayt2023}. Daher wird oft versucht durch eine externen Einspannung auf Systemebene diesen Druck aufzubringen. Jedoch entsteht durch die innere Reibung der Batterien kein gleichmäßiger Druckverlauf, was dazuführt, dass äußere Zellen stärker belastet werden und weiter innen liegende Zellen kaum von dem äußeren Druck profitieren. Auch haben höhere Ausgleichsdrücke, dass Problem, dass diese eine höhere Anstrengung für das Gesamtpaket darstellen, was zu dickeren Materialien und damit einer niedrigeren Gesamtenergiedichte führt.
Einzig die Knopfzellen, die durch eine integrierte Feder einen definierten Druck auf eine, im Verhältnis zur Pouchzelle, deutlich kleinere Fläche auswirkt ist die einzige bekannte Lösung zu diesem Problem. Hinzukommt, dass auch hier der Massenanteil von Gehäuse zu Zelle deutliche höher ist als bei Pouchzellen.

Für Strukturbatterien sind bisher keine Alternativen zum herkömmlichen Aluminiumpouchfolie untersucht wurden \cite{Ye2024}. Jedoch gibt es viele Gruppen die ihre Strukturbatterien mit Pouchfolie zusätzlich in einen kohlefaserverstärkten Kunststoff einbetten \cite{Pattarakunnan2020,Asp2021}. 


\section{Aktuelle Ansätze zur Entwicklung und Auslegung von Strukturbatterien}

\section{Ungelöste Herausforderungen in der Entwicklung von Strukturbatterien}
Erstellung Bild siehe Kommentar in .tex Datei
%Bild in Inkscape erzeugt und als SVG sowie pdf_tex speichern (Speichern unter -> .pdf -> Text in PDF weglassen und LaTex Datei erstellen). 


\begin{figure}[h]
	%\raggedleft
		%\def\svgwidth{\columnwidth}
	\def\svgscale{0.98}
		\input{testbild.pdf_tex} 
		\caption{\label{fig:testbild}Testbild erzeugt mit Inkscape}
\end{figure}

Bild \ref{fig:testbild} %\cite{Dannemann.Kucher_et.al_AppliedSciences_2018}  

Verwendung Package SIUNITX %(siehe Datei latex_package_readme_siunitx.pdf)   

Anzugsdrehmoment von $M_{\textnormal{a}}=\SI{1.1}{\newtonmetre}$

von \SI{1}{\kilo\hertz} bis \SI{15}{\kilo\hertz}

mittlere Temperaturänderung von $\left\langle \Delta T_{\textnormal{p}}\right\rangle(t)<\SI{0.5}{\degreeCelsius}$

Masse von $m_{\textnormal{p}}=\SI[separate-uncertainty]{0.884 (15)}{\gram}$

(siehe Abschnitt \ref{ch:anhang})
\chapter{Beschreibung der gekoppelten mechanischen-elektrochemischen Eigenschaften von Strukturbatterien}

\section{Physikalische Beschreibung der wichtigsten batteriechemischen Effekte}


\begin{equation}
    \nabla \cdot \boldsymbol{i}_{\text{s}} = \nabla \cdot \left( - \sigma \cdot \nabla \phi_{\text{s}} \right) = 0
\end{equation}

\begin{equation}
    \frac{\partial c_{\text{s}}}{\partial t}  = \nabla \cdot \left( D_{\text{s}} \nabla c_{\text{s}} \right) = 0
\end{equation}


\begin{equation}
    \frac{\partial c_{\text{s}}^{\pm}}{\partial t}(x,r,t) = \frac{1}{r^2} \frac{\partial}{ \partial r} \left[ D_{\text{s}}^\pm r^2 \frac{\partial c_{\text{s}}^\pm}{\partial r}(x,r,t)\right]
\end{equation}
mit den Randbedingungen
\begin{align}
    \frac{\partial c_{\text{s}}^{\pm}}{\partial r}(x,0,t) &= 0 \\
    \frac{\partial c_{\text{s}}^{\pm}}{\partial r}(x,R_{\text{p,s}}^{\pm},t) &= -\frac{1}{ D_{\text{s}}^\pm} j_{n}^{\pm}(x,t)
\end{align}
und
\begin{equation}
j_{n}^{\pm}(t) = \mp \frac{I(t)}{F a^{\pm} L^{\pm}}
\end{equation}


\begin{equation}
    C_{\text{A}}
\end{equation}


\section{Physikalische Beschreibung der wichtigsten mechanischen Effekte}
\subsection{Vereinfachte Modellierung der Gesamtsteifigkeit mittels klassischer Laminat-Theorie}


\subsection{Modellierung der Biegesteifigkeit unter Berücksichtigung des Elektrolyteffektes}

Das mechanische




\begin{equation}
    \sigma = E \varepsilon
\end{equation}

\begin{equation}
    F = A \sigma
\end{equation}

\begin{equation}
    \frac{\partial F}{\partial x}+ q_x(x) = 0
\end{equation}

Unter der Annahme eines linearen Deformationsverlaufes in einem Stab ergibt sich die folgende Verschiebungsfunktion 
\begin{equation}
    u_x(x) = \begin{bmatrix} 1-\frac{x}{l} & \frac{x}{l} \end{bmatrix}\begin{bmatrix} 
        u_{x,1} \\
        u_{x,2} 
    \end{bmatrix}.
\end{equation}

Mittels der Beziehung zwischen Dehnung und Deformation folgt 
\begin{equation}
    \varepsilon_x = \frac{\partial u_x}{\partial x} = 
    \frac{1}{l}
    \begin{bmatrix} 
        \text{-}1 & 1 
    \end{bmatrix}
    \begin{bmatrix} 
        u_{x,1} \\
        u_{x,2} 
    \end{bmatrix}.
\end{equation}

Aus dem Zusammenhang zwischen Normalspannung und Dehnung
\begin{equation}
    \sigma_x = E \varepsilon_x = 
    \frac{E}{l}
    \begin{bmatrix} 
        \text{-}1 & 1 
    \end{bmatrix}
    \begin{bmatrix} 
        u_{x,1} \\
        u_{x,2} 
    \end{bmatrix}
\end{equation}
kann eine Beziehung zur äußeren Kraft aufgestellt werden
\begin{equation}
    F = A \sigma = 
    \frac{EA}{l}
    \begin{bmatrix} 
        \text{-}1 & 1 
    \end{bmatrix}
    \begin{bmatrix} 
        u_{x,1} \\
        u_{x,2} 
    \end{bmatrix}
\end{equation}

\begin{align}
   - \frac{\partial u_x}{\partial x} \left( EA\; \frac{\partial u_x}{\partial x} \right) &= q_x[x]\\
    -EA\; \frac{\partial u_x}{\partial x} &= F
\end{align}

\begin{equation}
   [K_{\text{Stab}}] \boldsymbol{u}_{Stab} = \frac{E \; A}{l} 
   \begin{bmatrix}
    1 & \text{-}1 \\
    \text{-}1 & 1
   \end{bmatrix}
   \begin{bmatrix}
    u_{x,1} \\
    u_{x,2}
   \end{bmatrix}
   = 
   \begin{bmatrix}
    f_{x,1} \\
    f_{x,2}
   \end{bmatrix}
\end{equation}

Wegen der vergleichsweise geringen Höhe der Elektrode zu ihrer Länge können Schubverformungen vernachlässigt werden. Dies bedeutet, dass 


Durch de Minimierung der potentiellen Energie folgt
\begin{equation}
    [k] = \int_0^l [B]^T \; EI \; [B] dx
\end{equation}

\begin{equation}
    [K_{\text{Balken}}] \boldsymbol{u}_{Balken} = \frac{E \; I}{l^3} 
    \begin{bmatrix}
        12 & 6 \; l & \text{-}12 & 6 \; l        \\
        6 \; l & 4 \; l^2 & \text{-}6 \; l & 2 \; l^2 \\
        \text{-}12 & \text{-}6 \; l & 12 & \text{-}6 \; l      \\
        6 \; l & 2 \; l^2 & \text{-}6 \; l & 4 \; l^2
    \end{bmatrix}
    \begin{bmatrix}
        u_{y,1}  \\
        \Theta_1 \\
        u_{y,2}  \\
        \Theta_2
    \end{bmatrix}
    = 
    \begin{bmatrix}
        \frac{q \; l}{2}  \\
        \frac{q \; l^2}{12} \\
        \frac{q \; l}{2}  \\
        \text{-}\frac{q \; l^2}{12}
    \end{bmatrix} 
 \end{equation}

 \begin{align}
    [K_{\text{Elektrode}}] \boldsymbol{u}_{Träger} &= ([K_{\text{Stab}}] + [K_{\text{Balken}}])\boldsymbol{u}_{Elektrode}\\
    &= 
    \begin{bmatrix}
        \frac{E \; A}{l} & 0     & 0     &  \text{-}\frac{E \; A}{l}  & 0 & 0 \\
        0 & 12 \frac{E \; I}{l^3}     & 6 \frac{E \; I}{l^2} & 0    &\text{-}12\frac{E \; I}{l^3}  & 6 \frac{E \; I}{l^2}       \\
        0 & 6 \frac{E \; I}{l^2} & 4 \frac{E \; I}{l}  & 0    & \text{-}6 \frac{E \; I}{l^2}  & 2 \frac{E \; I}{l} \\
        \text{-}\frac{E \; A}{l} & 0     & 0     &  \text{-}\frac{E \; A}{l}  & 0 & 0 \\
        0 & \text{-}12\frac{E \; I}{l^3}    & \text{-}6 \frac{E \; I}{l^2}& 12\frac{E \; I}{l^3}  & 0    & \text{-}6 \frac{E \; I}{l^2}      \\
        0 & 6 \frac{E \; I}{l^2} & 2 \frac{E \; I}{l} & 0    & \text{-}6 \frac{E \; I}{l^2} & 4 \frac{E \; I}{l}
    \end{bmatrix}
    \begin{bmatrix}
        u_{x,1}  \\
        u_{y,1}  \\
        \Theta_1 \\
        u_{x,2}  \\
        u_{y,2}  \\
        \Theta_2
    \end{bmatrix}
    = 
    \begin{bmatrix}
        f_{x,1} \\
        \frac{q \; l}{2}  \\
        \frac{q \; l^2}{12} \\
        f_{x,2} \\
        \frac{q \; l}{2}  \\
        \text{-}\frac{q \; l^2}{12}
    \end{bmatrix} 
 \end{align}



Das mechanische Verhalten des porösen Aktivmaterials unterscheidet sich besonders bei Kompression. Qu et al.  schlagen zur Modelierung das Deshpande-Fleck Schaummodel \cite{Deshpande2000} vor, welches vorher bereits zur Beschreibung von Gesteinsschichten und Metallschäumen zum Einsatz kam \cite{Qu2023}.
\begin{equation}
    F = \sqrt{q^2 + \alpha^2 (p-p_0)^2} - B = 0
\end{equation}

Wenn sich Elektrolyte zwischen den Elektroden befindet entsteht durch die Deformation eine Volumenänderung, die über die ideale Gasgleichung eine Veränderung im innen Druck bewirkt.

\begin{equation}
    p\;V = n \; R_{K} \ T
\end{equation}
Bei konstanter Temparatur kann die Zustandsänderung des Systems mit Boyles Gesetz beschrieben werden. 
\begin{equation}
    p_1 \; V_1 = p_2 \; V_2
\end{equation}
Bei iener zeitlichen Diskretisierung ist allerdings eine leicht modifizierte Schrebiweise hilfreicher.
\begin{equation}
    p_{t} \; V_{t} = p_{t+\Delta t} \; V_{t+\Delta t}
\end{equation}

\begin{equation}
    p_{t+\Delta t}   = \frac{p_{t} \; V_{t}}{V_{t+\Delta t}}
\end{equation}

Der Druckausgleich im System wird durch die zahlreichen Widerstände verhindert, was durch die Naviar-Stokes-Gleichung beschrieben wird.
\begin{equation}
    \rho \frac{\partial \boldsymbol{u}}{\partial t} + \rho \boldsymbol{u} \cdot \nabla \boldsymbol{u} = -\nabla p + \mu \nabla^2\boldsymbol{u} + f_{ext}
\end{equation}
Für inkompressible Flüssigkeiten gilt außerdem
\begin{equation}
    \nabla \cdot \boldsymbol{u} = 0.
\end{equation}
Die Gleichung hat fünf Terme. Dabei beschreiben die ersten Beiden den Einfluss der Trägheit uaf das Systems, der Dritte den Druckgradient, der Vierte die Vikosität und der Fünfte die Externen Kräfte. 

\begin{equation}
    \frac{\partial \boldsymbol{v}}{ \partial t} = - \nabla p + \frac{1}{\text{Re}} \nabla^2\boldsymbol{v} + \boldsymbol{f}
\end{equation}

Der Volumenstrom wiederum hat 

\section{Koppelung der mechanischen und elektrochemischen Effekte}


\section{Fehlerabschätzung der Modellierung durch Literturwerte}

\section{Zusammenfassende Darstellung der Modellierungsmethodik}
\chapter{Effizientere digitale Auslegung von Strukturbatterien}
\section{Konzeptionierung eines Design und Auslegungstools zur vereinfachten Entwicklung von Strukturbatterien}

\section{Erstellung einer Materialdatenbank für Strukturbatteriekomponenten}

\section{Programm Schnittstellen}

\section{Koppelung der Auslegungsmethodiken}

\section{Effiziente Identifizierung geeigneter Strukturbatterien}

\section{Umsetzung der Auslegungsmethodik}


\chapter{Ergebnisse}
\section{Validierung der Eigenschaftsvorhersagen}

\section{Ashby Logic}

\section{Design of Recyability}

\chapter{Mögliche Anwendungen von Strukturbatterien im Leichtbau}
\section{Roadmap}
\section{Potenzial für elektrische Fahrräder}

\chapter{Abschließende Bemerkungen}

\section{Zusammenfassung und Bewertung}

Die 

\begin{itemize}
    \item Digitale Entwicklungsmethodik für die gezielte Entwicklung von Strukturbatterien mit 
    \item Eigenschaftspotenziale von herstellbaren Strukturbatterien identifiziert und abgeleitet
    \item Modellierung der mechanischen und elektrochemischen Eigenschaften unter Berücksichtigung des Strukturelektrolyteffektes
    \item Nachweis einer ersten Modellbasierten mittels prototypischen Strukturbatterie erbracht
    \item Transfer der Strukturbatterien in potentielle Anwendungen gezeigt
\end{itemize}

\section{Ausblick}

Die

%Literaturverzeichnis und Datenbank einfügen
\nocite{} %\nocite{*} --> alle in der Datenbank existierenden Einträge werden bearbeitet; ohne * --> nur die verwendeten werden aufgeführt
\bibliographystyle{plaindin_mod} %Aussehen des Literaturverzeichnisses
\bibliography{Literatur_Diss} % Einbinden der Literaturdatenbank <yyyymmdd_Literatur.bib>
%\printbibliography
\appendix
\chapter{\label{ch:anhang} Anhang}





\cleardoublepage
\pagestyle{empty}
\section*{Lebenslauf}

\begin{tabbing}
\hspace*{4cm}\= \kill

\textbf{Persönliche Daten}\\[0.45em]%
Name							\>  Willi Zschiebsch	\\[0.45em]
Geburtsdatum					\> 	28.10.1996			\\[0.45em]
Geburtsort						\> 	Wurzen  			\\[0.45em]
Eltern							\> 						\\
								\> 						\\[0.45em]
Familienstand					\> 	ledig				\\[0.45em]
Kinder      					\> 	keine				\\[0.45em]
            					\> 						\\[0.45em]

Staatsangehörigkeit		\> 		deutsch					\\[1.5em]

\textbf{Ausbildung} \\[0.45em]
2015 - 2018	\> Maschinenbau Master an der HTWK Leipzig	\\[0.5em]	
2018	\> Forschungsaufenthalt im Robotiklabor von Prof. Amir Shapiro 	\\[0.5em]
2015 - 2018	\> Maschinenbau Bachelor an der HTWK Leipzig	\\[0.5em]
2015	\> Abitur Wilhelm-Ostwaldgymnasium Leipzig	\\[0.5em]
															\\[1.5em]

\textbf{Auszeichnungen und Stipendien} \\[0.45em]
03/2015 \> Sonderpreis des Ministeriums für Wirtschaft, Arbeit und Verkehr, Sachsen \\[0.5em]
06/2015 \>  Preis für eine besondere Leistung im Gebiet der Technik von Heinz und \\ 
		\> Gisela Friederichs Stiftung \\[0.5em]
06/2015 \>  Qualifizierung und Teilnahme für den Bundeswettbewerbes „jugendForscht“ \\[0.5em]
12/2015–12/2020 \>  Mitglied in der Studienstiftung des deutschen Volkes \\[0.5em]
10/2018 \> Preis der Karl-Kolle-Stiftung \\[0.5em]
06/2019 \> Dritter Platz des VDI-Förderpreises \\[0.5em]
10/2021 \> Preis der Karl-Kolle-Stiftung \\[0.5em]
06/2021-05/2024 \>  HTWK-Promotionsstipendium \\[0.5em]
04/2022 \> Best Speaker-Award auf der ICOMS21 \\[0.5em]
09/2022 \> Best Presentation Award auf der ICAMDS 2023 \\[0.5em]

\end{tabbing}





%%Diplom-\\
%%praktikum:\> 1994\> Institut f"ur Theoretische Physik\\
%%\>\>(Abt. B) der TU Clausthal\\
%wiss. Hilfskraft:\> 1993 - 1994\> Physikalisches Institut (Abt. Experimentalphysik)\\
%\>\> der TU Clausthal\\
%\> 1995\>  Institut f"ur Theoretische Physik\\
%\>\> (Abt. B) der TU Clausthal\\
%\> 1996 - 1998\> Institut f"ur Leichtbau und Kunststofftechnik\\
%\>\> der TU Dresden\\
%Stipendien:\> 1997 - 1998\> Landesinnovationsstipendium Sachsen (LIST)\\
%wiss. Mitarbeiter:\> seit 1998\> Institut f"ur Leichtbau und Kunststofftechnik\\
%\>\> der TU Dresden
%\end{tabbing}

\end{document}

