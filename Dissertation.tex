%**************************************************************************
%**************************************************************************
%*********** DOKUMENTDEKLARATION und GRUNDFORMATIERUNG ********************
%**************************************************************************
%**************************************************************************
\documentclass[
draft=false, % Entwurfsmodus an/aus true|false
DIV=14, % DIVzahl|DIV=calc - Satzspiegel (Seitenränder), ggf. geometry-Paket nutzen
fontsize=12pt, % Schriftgröße
a4paper, % Papierformat a5paper|a4paper
twoside, % Ein- oder zweiseitig oneside|twoside
numbers=noendperiod,
numbers=noenddot, % kein Punkt nach Abschnitten (statt 3.2.1. also 3.2.1)
toc=bibliography, % Literaturverzeichnis ins Inhaltsverzeichnios aufnehmen
parskip=half, %Absatzformat ohne Erstzeileneinzug, Abstand zwischen Absätzen
headings=normal, % Größe der Überschriften small|normal|big
%ngerman, % Neue Deutsche Rechtschreibung
open=any, % any|right - Kapitel fangen auf einer neuen Seite (any) an oder immer reechts (right)
headinclude=true, % Kopfzeile ist Teil des Satzspiegels
footinclude=false, % Fußzeile ist nicht Teil des Satzspiegels
BCOR=10mm, % Breite des Bundsteg
pagesize=auto, % Ausgabetreiber auto|pdftex|dvips|dvipdfmx|false|automedia
]{scrreprt} % [bibgerm] % report oder KOMA: scrreprt|scrbook

\usepackage[
headsepline, % Linie zwischen Kopfzeile und Text %
%footsepline,
autooneside, % 
automark, %
]{scrlayer-scrpage} % Für die Formatierung von Kopfzeilen

%\usepackage{showframe} % Zeigt den Satzspiegel an
%\usepackage{layouts}

\usepackage[ngerman]{babel}

%\usepackage{tocloft}
\usepackage{scrhack} % Verbessert das Zusammenspiel zwischen KOMA und dem float-Paket

\usepackage[utf8]{inputenc} %Korrekte Codierung Umlaute etc.
\usepackage{xspace}


\usepackage[T1]{fontenc} %Nur ändern, wenn kyrillische Gedichte oder japanische Kanji zu setzen sind...
%%%\usepackage{geometry} % Zum manuellen Einstellen der Seitenränder etc.
%%%\geometry{
%%%twoside=false, % für zweiseitiges Layout
%%%includehead, % Kopfzeile ist Teil des Satzspiegels
%%%%includefoot, % Kopfzeile ist Teil des Satzspiegels
%%%inner=24mm,
%%%outer=24mm,
%%%top=24mm,
%%%bottom=35mm,
%%%bindingoffset=10mm, % Breite des Bundsteg
%%%}

%**************************************************************************
%**************************************************************************
%*********** PAKETDEKLARATION *********************************************
%**************************************************************************
%**************************************************************************

%------ MK global path for input -----------------------------------
\makeatletter
\def\input@path{
{Abbildungen/01_Einleitung/}
{Abbildungen/02_Kapitel/}
{Abbildungen/03_Kapitel/}
{Abbildungen/04_Kapitel/}
{Abbildungen/05_Kapitel/}
{Abbildungen/06_Kapitel_Zusammenfassung/}
{Abbildungen/09_Anhang/}}
%or: \def\input@path{{/path/to/folder/}{/path/to/other/folder/}}
\makeatother

%------ MikTeX Standardpakete --------------------------------------
\usepackage[pdftex]{graphicx} \DeclareGraphicsExtensions{.pdf,.jpg}
\graphicspath{
{Abbildungen/01_Einleitung/}
{Abbildungen/02_Kapitel/}
{Abbildungen/03_Kapitel/}
{Abbildungen/04_Kapitel/}
{Abbildungen/05_Kapitel/}
{Abbildungen/06_Kapitel_Zusammenfassung/}
{Abbildungen/09_Anhang/}
} % Definiere Grafikpfade

\usepackage{lmodern}

\usepackage{latexsym}     % Definiert einige Latex-Sonderzeichen (Package ist Teil von LaTeX 2e)
\usepackage{bibgerm} %bibgerm:	Deutsches Literaturverzeichnis

\usepackage{multirow}          % enthält die Funktion eine Tabellenzelle über mehrer Zeilen zu verbinden

%----- persönl. Anpassungen ----------------------------------------
\usepackage{diss_style}				% Anpassungen mda

\usepackage{paralist}					 % für kompakte Aufzählungen (compactitem oder compactenum)
\usepackage{color}						 % für farbigen Text
\usepackage{threeparttable}		 % für Tabellen mit z.B. Fußnoten etc.
\usepackage{ragged2e}					 % für \RaggedRight u.ä. mit Silbentrennung
\usepackage{slashbox}					 % Diagonale Linien in Tabellenzelle
\usepackage{calc, dcolumn}		 % für Ausrichtung am Dezimalpunkt in Tabellen
\usepackage{amsmath}
\usepackage{amssymb}
\usepackage{amsfonts}
\usepackage{nicefrac} % Darstellung von schrägen Bruchstrichen, z.B. E'/E'' im Fließtext
\usepackage{float}
\usepackage[section]{placeins} % z.B. \FloatBarrier-Befehl zur Ausgabe aller Gleitobjekte ohne Seitenumbruch (Alternative zu \clearpage)
															% Option [section] erzwingt nach jeder Section einen \FloatBarrier-Befehl, d.h. keine Float-Objekte dürfen in die nächste Section "rutschen"
\usepackage{longtable}			% Tabellen über mehrere Seiten (z.B. beim Symbolverzeichnis)
\usepackage{textcomp} % mk \textmu
\usepackage{rotating} % mk
\usepackage{pdfpages} % mk: Einbinden von PDF-Dokumenten
\usepackage{listings} % mk: EInbindung Matlab typeset
% Für das siunitx müssen die aktuellen Pakete l3kernel und l3packages über den MikTex-Package Manager installiert werden!
\usepackage{siunitx} % für Einheiten [version-1-compatibility]
\sisetup{
list-units = single,
range-phrase = \,\ldots,
exponent-product=\cdot,
per-mode=fraction,
output-decimal-marker={,},
detect-family,
detect-all, % Detektiert die aktuelle Schriftart
%text-rm=\normalfont, % wenn SI{} im Text auftauch als normalen Text setzen
%binary-units = true
} 
\DeclareSIUnit{\wtpercent}{wt\%}
\DeclareSIUnit{\newtonmetre}{Nm}
\DeclareSIUnit{\tanDeltaMilli}{10^{-3}}

\usepackage{hvfloat} % Für Querformatobjekte, die nicht übner die ganze Seite gehen

%\usepackage[color]{changebar} % Für seitliche Striche, insb. zur Hervorhebung von Änderungen
%	\cbcolor{red}
%	\setlength{\changebarwidth}{5pt}

\usepackage{enumitem} % zur Formatierung von Aufzählungen
\usepackage{tikz} % Für Text in Kreisen, z.B. für Aufzählungen

%\usepackage[
%pict2e, %
%ngerman, %
%final, %draft|final
%]{struktex} % für Struktogramme (Algorithmenstrukturen)
%\AtBeginEnvironment{struktogramm}{\small} % Schrifteinstellungen für Struktogramme

%\AtBeginEnvironment{table}{\small}
%\AtBeginEnvironment{threeparttable}{\small}
%\AtBeginEnvironment{mtabular}{\small}

\usepackage{ifthen} % Für bedingte Anweisungen

% Für richtige Literaturverweise mit Links
\usepackage[numbers,sort&compress]{natbib}
\usepackage[pdftex,plainpages=false,pdfpagelabels=true,pdfnewwindow]{hyperref} % Hyperref so weit wie möglich hinten einbinden

% Für Koma-Zeitangaben (insb. für die Angabe von Datum in Zeit der Entwurfsversionen
%\usepackage{scrdate} % Funktion \today bereits in MikTex-Kernel enthalten
\usepackage{scrtime}

% Informationen für pdf-Dokument
\usepackage{hypcap}    % Damit in Bilder an die obere Kante gesprungen wird
\hypersetup{%
       pdfauthor={Vorname Nachname},
       pdftitle={Dissertation},
       pdfcreator={pdfTeX},
       pdfsubject={Titel der Dissertationsschrift},
       pdfkeywords={FKV, FVW, FVK, GFK, CFK, FEM, BEM, Bauteil, Auslegung, Messung},
       plainpages=false,
       bookmarksopen=true,
       bookmarksnumbered=true,
       bookmarksopenlevel=2,       
       plainpages=false,
			 pdfpagemode=FullScreen,
       pdfview=FitV,
       pdffitwindow=true,
       citecolor=blue,
       urlcolor=blue,
       pdfmenubar=true,
       pdfwindowui=true,
       pdfhighlight=/I,
       pdfborder={0 0 0},
       pdfpagemode=UseOutlines,
    	 pdfstartview=Fit,
			 linktocpage=true			
}
%\pdfminorversion=6 %Fehlermeldungen beim Einlesen von pdf-Bildern ausblenden, die mit pdf-Veriosn 1.5 und später erstellt wurden
\typeout{Anpassungen von mda, Stand 10.10.2012 (teilweise basierend auf TEXDEFS.TEX)}

\newboolean{Entwurfsmodus}
\setboolean{Entwurfsmodus}{true} % true|false

\definecolor{lightgray}{rgb}{0.5,0.5,0.5}
\definecolor{lightgrey}{rgb}{0.5,0.5,0.5}
\definecolor{darkred}{rgb}{0.6,0,0}

%\newcommand{\avg}[1]{\left< #1 \right>} % for average
\newcommand{\rot}[1]{\textcolor{red}{#1}} %rote Hervorhebung

% Breitenangabe fuer Strukturgramme
% 	Es ist leider nicht moeglich die Breite eines Strukturgramms auf die Textbreite zu beziehen, daher 150 fuer A4, und 105 fuer A5 einstellen
%		sProofOn muss trotzdem noch fuer jedes Diagramm wegen der Hoeheneinstellung durchgefuehrt werden!!!
\newcommand{\struktbreite}{150}

% Fuer eigene Verwendung
\ifthenelse{\boolean{Entwurfsmodus}}
{
\typeout{Entwurfsmodus an.}
\newcommand{\korr}[1]{\textbf{\textcolor{magenta}{#1}}} %rote Hervorhebung
\newcommand{\kommentar}[1]{ \textcolor{green}{\textit{(#1)}} } % Kommentar; in Endfassung einfach Inhalt der geschweiften Klammer loeschen
\newcommand{\fertig}{\texorpdfstring{ \textcolor[rgb]{0,0.8,0}{(\checkmark)}}{ o.k.}} % macht ein Haekchen innnherhalb von Ueberschriften
\newcommand{\inArbeit}{\texorpdfstring{ \textcolor[rgb]{0.8,0,0}{(\checkmark)}}{ i.A.}} % macht ein Haekchen innnherhalb von Ueberschriften
\newcommand{\fehlt}{\texorpdfstring{ \textcolor[rgb]{0.8,0,0}{(\textbf{!!})}}{ fehlt}} % macht ein Haekchen innnherhalb von Ueberschriften
\newcommand{\alt}[1]{\underline{ALT:} \textcolor{lightgray}{#1}} %alte Textpassagen
\newcommand{\neu}[1]{\underline{NEU:} \textcolor{darkred}{#1}} %neue Textpassagen
}{
%%%%% Fuer Ausdruck und Weitergabe
\typeout{Entwurfsmodus aus.}
\newcommand{\korr}[1]{} %rote Hervorhebung
\newcommand{\kommentar}[1]{} % Kommentar; in Endfassung einfach Inhalt der geschweiften Klammer loeschen
\newcommand{\fertig}{} % macht ein Haekchen innnherhalb von Ueberschriften
\newcommand{\inArbeit}{} % macht ein Haekchen innnherhalb von Ueberschriften
\newcommand{\fehlt}{} % macht ein Haekchen innnherhalb von Ueberschriften
\newcommand{\alt}[1]{} %alte Textpassagen
\newcommand{\neu}[1]{#1} %neue Textpassagen
}

%% Sonstige Definitionen
% Original war hinter Bild Tabelle Abschnitt immer ein ~ - also ein nicht trennbares Leerzeichen
\newcommand{\bild}[1]{Bild \ref{#1}} % Bildverweise
\newcommand{\kreis}[1]{\unitlength1ex\begin{picture}(2.5,2.5) \put(0.75,0.75){\circle{2.5}}\put(0.75,0.75){\makebox(0,0){#1}}\end{picture}} % Einekreiste Zahl oder Buchstabe
\newcommand{\tabelle}[1]{Tabelle \ref{#1}} % Tabellenverweise
\newcommand{\abschnitt}[1]{Abschnitt \ref{#1}} % Gliederungsverweise
\newcommand{\anhang}[1]{Anhang \ref{#1}} % Verweise zum Anhang
\newcommand{\name}[1]{\textsc{#1}} % Namentliche Nennungen
\newcommand{\fett}[1]{\textbf{#1}} % Fett
\newcommand{\ul}[1]{\underline{#1}} % unterstrichen --> in Paket soul bereits definiert
\newcommand{\dul}[1]{\underline{\underline{#1}}} % Doppelt unterstrichen
\newcommand{\mat}[1]{\pmb{#1}} % Kennzeichnung von Matrizen besser als \pmb (poor mans bold) waere \boldsymbol, welches allerdings leider nicht funktioniert -> fehlende Schriftart???
\newcommand{\kmplx}[1]{\ul{#1}} % Kennzeichnung komplexer Groessen
\newcommand{\kkmplx}[1]{\kmplx{#1}^\ast} % Kennzeichnung einer konjungiert komplexen Groesse
\newcommand{\oampl}[1]{\hat{#1}} % Ortsamplitude
\newcommand{\zampl}[1]{\check{#1}} % Zeitamplitude
\newcommand{\avrg}[1]{\overline{#1}} % Mittelwert
\newcommand{\gl}[1]{(\ref{#1})} % Gleichungsverweis
\newcommand{\UD}{\text{UD}}	% UD in Formeln (unidirektional)
\newcommand{\TV}{\text{TV}} % TV in Formeln (textilverstaerkt)
\newcommand{\EX}{\text{Exp.}} % Exp. in Formeln (experimentell ermittelt)
\newcommand{\FE}{\text{FE}} % FE in Formeln
\newcommand{\DS}{\text{DS}} % Deckschicht DS in Formeln
\newcommand{\KS}{\text{KS}} % Kernschicht DS in Formeln
\newcommand{\MAC}{\text{MAC}} % MAC in Formeln
\newcommand{\Terz}{\text{Terz}} % MAC in Formeln
\newcommand{\Wert}{\text{Wert}} % Formelzeichen Wert
\newcommand{\Geni}{{\text{Gen}_i}} % Gen_i in Formeln
\newcommand{\Kind}{\text{Kind}} % Kind in Formeln
\newcommand{\Elter}{\text{Elter}} % Elter in Formeln
\newcommand{\Lag}{\text{Lag}} % Lag fuer Lagerung in Formeln
\newcommand{\Mat}{\text{Mat}} % Mat fuer Material in Formeln
\newcommand{\DEVOP}{\name{DevOP}\xspace} % DEVOP- im Text

%-------------------------------------------------------------------
% Globale Trennvorschlaege
\hyphenation{Donau-dampf-schiff-fahrt}
\hyphenation{Ur-instinkt}
\hyphenation{ei-nen}
\hyphenation{vi-bro-akus-tisch}
\hyphenation{vi-bro-akus-tische}
\hyphenation{vi-bro-akus-tisch-en}
\hyphenation{vi-bro-akus-tisch-em}
\hyphenation{vi-bro-akus-tisch-es}
\hyphenation{vi-bro-akus-tisch-er}
\hyphenation{Vi-bro-akus-tik}
\hyphenation{Mehr-schicht-ver-bund}
\hyphenation{Straf-funk-tion-en}
\hyphenation{an-iso-trop}
\hyphenation{an-iso-tro-pe}
\hyphenation{an-iso-tro-pen}
\hyphenation{an-iso-tro-pem}
\hyphenation{an-iso-tro-per}
\hyphenation{An-iso-tro-pie}
\hyphenation{uni-di-rek-tio-nal}
\hyphenation{uni-di-rek-tio-nale}
\hyphenation{uni-di-rek-tio-nal-en}
\hyphenation{Leicht-bau-struk-tur-en}
\hyphenation{vis-ko-elas-tisch}
\hyphenation{be-an-spruch-ten}
\hyphenation{Stich-pro-ben-grö-ße}
\hyphenation{me-si-o-buk-kal}
\hyphenation{pa-la-tal}
\hyphenation{ste-re-o-mi-kros-ko-pi-sche}

%-------------------------------------------------------------------
%% Ligaturen - leider nicht global einstellbar
%% ff fi fl ffi ffl ll lll
% Schall"|leistung
% werkstoff"|inhärent
% Werkstoff"|integration


%-------------------------------------------------------------------
%\newcommand{\entspr}{\widehat{=}}
\newcommand{\entspr}{\stackrel{\scriptscriptstyle\wedge}{=}}
\newcommand*{\dunderl}[1]{\underline{\underline{#1\!}}\,}
\renewcommand{\Re}{\mbox{Re}} %normalerweise: Schn"orkel-R
\renewcommand{\Im}{\mbox{Im}} %normalerweise: Schn"orkel-I
\renewcommand{\i}{{\rm i}}    %normalerweise: i ohne Punkt
\newcommand{\de}{{\mathsf d}}  %sans serif für elektrische Groessen
\newcommand{\e}{{\mathsf e}}
\newcommand{\E}{{\mathsf E}}
\newcommand{\U}{{\mathsf U}}
\newcommand{\T}{{\rm T}} % Transponiert-Operator
\newcommand{\N}{\mbox{I\hspace{-0.19em}N}} % Menge N = nat"urliche Zahlen
\newcommand{\R}{\mbox{I\hspace{-0.19em}R}} % Menge R = reelle Zahlen
\newcommand{\C}{\mbox{\hspace{0.35em}\rule{0.04em}{1.55ex}\hspace{-0.35em}C}} % Menge C = komplexe Zahlen
\newcommand{\mitDelta}{\mathnormal{\Delta}}  % neu: LaTeX 2e
\newcommand{\mitTheta}{\mathnormal{\Theta}}
\newcommand{\mitPhi}{\mathnormal{\Phi}}
\newcommand{\mitPsi}{\mathnormal{\Psi}}
\newcommand{\brmitTheta}{\breve{\mitTheta}}
\newcommand{\brmitPhi}{\breve{\mitPhi}}
\newcommand{\brmitPsi}{\breve{\mitPsi}}
\newcommand{\brrho}{\breve{\rho}}
% fuer Buchstaben mit Kreisen
\newcommand*\mycirc[1]{%
\begin{tikzpicture}[baseline=(C.base)]
\node[draw,circle,inner sep=1pt,minimum size=3ex](C) {#1};
\end{tikzpicture}}

%\renewcommand{\theta}{\fehlertheta}      % \
%\renewcommand{\theta}{\vartheta}
\renewcommand{\phi}{\varphi}
%\renewcommand{\phi}{\fehlerphi}          %  zur Sicherheit
%\renewcommand{\epsilon}{\textepsilon}  % /
%\renewcommand{\epsilon}{\fehlerepsilon}  % /
% Das Symbol \phi (hier \varvarphi) wird fuer Sonderfaelle benoetigt.
% Folgende Definition entstammt fontmath.ltx.
\DeclareMathSymbol{\varvarphi}{\mathord}{letters}{"1E}
\DeclareMathSymbol{\eps}{\mathsf}{letters}{"0F} 
    
% Neue Spaltenarten für Tabellen
\newcolumntype{C}[1]{>{\centering\arraybackslash}p{#1}} % Mehrzeilige horizontal zentrierte Tabellenspalten
\newcolumntype{M}[1]{>{\centering\arraybackslash}m{#1}} % Mehrzeilige horizontal und vertikal zentrierte Tabellenspalten
%\newcolumntype{V}[1]{>{\RaggedRight\arraybackslash}m{#1}} % Mehrzeilige linksbündige und vertikal zentrierte Tabellenspalten
\newcolumntype{L}[1]{>{\RaggedRight\arraybackslash}p{#1}} % Mehrzeilig linksbündige oben ausgerichtete Spalten
\newcolumntype{R}[1]{>{\RaggedLeft\arraybackslash}p{#1}} % Mehrzeilig linksbündige oben ausgerichtete Spalten
\newcolumntype{d}{D{.}{,}{-1}}													% Ausrichtung am Dezimalpunkt und Satz als Komma
\newcommand{\tabkopf}[1]{\textbf{#1}}
\newcommand{\Dkopf}[1]{\multicolumn{1}{c}{\textbf{#1}}}% tabkopf für Spaltentyp D
\newcommand{\DkopfZU}[2]{\multicolumn{1}{M{#1}}{\textbf{#2}}} % tabkopf mit Zeilenumbruch
\newcommand{\DZU}[2]{\multicolumn{1}{M{#1}}{#2}} % tabkopf mit Zeilenumbruch
\newcommand{\Dtext}[1]{\multicolumn{1}{c}{#1}} 				% normaler Text im Spaltentyp D
\newcommand{\DSkopf}[2]{\multicolumn{1}{M{#1}|}{\textbf{#2}}}% tabkopf für Spaltentyp S mit Umbruchmöglichkeit
\newcommand{\DStext}[2]{\multicolumn{1}{M{#1}|}{#2}} 				% normaler Text im Spaltentyp S mit Umbruchmöglichkeit
% Einfacher Abstand in Tabellenumgebung mtabular
\newenvironment{mtabular}{%
	\small%
  \renewcommand*{\arraystretch}{1.4}%
  \tabular
}{%
  \endtabular
}

\pdfsuppresswarningpagegroup=1 %mk Unterdrueckung warning of multiple PDF files

\usepackage{lscape} %mk Tabelle Querformat
\usepackage{amsmath} %mk for boldsymbol

\usepackage[percent]{overpic} %mk Bild auf ganzer Seite

\ifthenelse{\boolean{Entwurfsmodus}}{
	\hypersetup{%
			colorlinks=true, % true|false
      linkcolor=blue,	% blue|black       
      citecolor=blue,		% blue|black
	}
} {
	\hypersetup{%
			colorlinks=false, % true|false
      linkcolor=black,	% blue|black       
      citecolor=black,		% blue|black
	}
}

%**************************************************************************
%**************************************************************************
%*********** BEGINN des Dokumenteninhaltes ********************************
%**************************************************************************
%**************************************************************************
\begin{document}

\pagenumbering{gobble} % Ausschalten der Seitenzählung für die Titelseite und das Vorwort

\begin{titlepage}

\centering
\vspace*{-2.0cm}
%\vfill

{\renewcommand{\baselinestretch}{1.4}\LARGE
{\textbf{Titel}}\par
}

\vspace{1cm}

Von der Fakultät Maschinenwesen \\[+0.5\baselineskip]
der\\[+0.5\baselineskip]
Technischen Universität Dresden \\[+0.5\baselineskip]
zur\\[+0.5\baselineskip]
Erlangung des akademischen Grades \\[+0.5\baselineskip]
Doktoringenieur (Dr.-Ing.) \\[+0.5\baselineskip]
angenommene Dissertation

\vspace{1cm}

von \\[+0.5\baselineskip]
{\large M. Eng.\ Willi Zschiebsch} \\[+0.5\baselineskip]
geboren am 28. Oktober 1996 in Wurzen \\

\vspace{1cm}

\begin{tabbing}
\hspace*{4.3cm}\= \kill
Tag der Einreichung: \> TT.MM.JJJJ \\
Tag der Verteidigung: \> TT.MM.JJJJ 
\end{tabbing}

\vspace{1cm}

\raggedright
Promotionskommission:
\begin{tabbing}
\hspace*{2.8cm}\= \kill
Vorsitzender: 	\> Prof. habil. Dr.-Ing. Maik Gude\dots \\[+0.5\baselineskip]
Gutachter:    	\> Prof. Dr.-Ing. Niels Modler \dots \\
				\> Prof. habil. Dr.-Ing. Robert Böhm \\[+0.5\baselineskip]
Beisitzer: 		\> Prof. \dots \\
							\> Prof. \dots \\
\end{tabbing}
\end{titlepage}


%--------------------------------------------------------------------------
\hypersetup{pageanchor=false}
\thispagestyle{empty}
\cleardoublepage

\thispagestyle{empty}

\vspace*{\fill}\vspace*{\fill}\vspace*{\fill}\vspace*{\fill}\vspace*{\fill}

\subsection*{Vorwort}

Die vorliegende Arbeit entstand während meiner Zeit als Promotionsstipendiat in der Arbeitsgruppe Leichtbau mit Verbundstoffen unter der Leitung von Prof. Dr.-Ing. habil. Robert Böhm. Finanziert wurde die Arbeit von der HTWK im Rahmen des HTWK-Promotionsstipendiums.

Meinem hochverehrten Doktorvater, Herrn Prof. Dr.-Ing. Niels Modler vom Institut für Leichtbau und Kunststofftechnik der Technischen Universität Dresden, möchte ich meinen besonderen Dank für die kontinuierliche Unterstützung ausdrücken.

Hervorzuheben ist die Unterstützung meiner Kollegen im ElViS-Projekt, insbesondere Johannes Kühn, Robert Seidel-Greiff, Daniel Wolz und Thomas Behnisch, für die vielen fachlichen Diskussionen, Hinweise und ihren großen Einsatz, der in die experimentelle Validierung der in dieser Arbeit entwickelten Methode geflossen ist.

Großer Dank gebührt auch meinen vielen Kollegen, die mir während meiner Forschungsarbeit motivierend zur Seite standen. Besonders hervorzuheben sind die Beiträge von Philipp Johst, Davood Peyrow Heyadati, Saskia Roßberg und Peter Jakob.

Abschließend bedanke ich mich sehr herzlich für den bedingungslosen Beistand und die Nachsicht meiner Freunde und meiner Familie, die für mich einen wichtigen Rückhalt und Ausgleich bedeuteten.

\cleardoublepage
\hypersetup{pageanchor=true}
\hypersetup{pageanchor=false}
\chapter*{Kurzfassung}
\thispagestyle{empty}

Das

\textbf{Schlagworte:} 

\vspace{1cm}
{\bfseries \LARGE Abstract}
\vspace{1cm} 

The

\textbf{Keywords:} 

\cleardoublepage
\hypersetup{pageanchor=true}

\pagenumbering{roman} % Römische Zahlen für den ganzen Vorspann

\pdfbookmark[1]{\contentsname}{toc}	% Link zum Inhaltsverzeichnis in der pdf-Datei
\tableofcontents

%\clearpage
%\listoffigures % Abbildungsverzeichnis
%
%\clearpage
%\listoftables % Tabellenverzeichnis

\clearpage
%\chapter*{Symbolverzeichnis}
%\label{sec:Symbolverzeichnis}
\addchap{Symbolverzeichnis}
\markboth{Symbolverzeichnis}{Symbolverzeichnis}


%Einzug erste Spalte
\newlength{\TabulatorVZ} % definiert einen neuen L�ngenparameter
\settowidth{\TabulatorVZ}{$m$, $n$, $i$, $\imath$, ${\bar{\imath}}$, $\jmath$\quad} % Setzt den L�ngenparameter auf den Wert, den der Text hat
% Einzug Einheitenspalte
\newlength{\TabulatorEH} % definiert einen neuen L�ngenparameter
\setlength{\TabulatorEH}{\widthof{$\si{\celsius}$; $\si{\kelvin}$}}

\newlength{\TabulatorTX}
\setlength{\TabulatorTX}{\textwidth}
\addtolength{\TabulatorTX}{-\TabulatorVZ-\TabulatorEH-2\tabcolsep}

{\renewcommand*{\arraystretch}{1.2}%

\section*{Abkürzungen}

\begin{longtable}{@{}p{\TabulatorVZ}@{}p{\TabulatorTX+\TabulatorEH+2\tabcolsep}@{}}
FE								& Finite Elemente \\
FEM								& Finite"=Elemente"=Methode \\
PA/PA6						& Polyamid/Polyamid-6 \\
PEEK    	        & Polyetheretherketon \\
PP								& Polypropylen

\end{longtable}

\section*{Allgemeine Notation}

\begin{longtable}{@{}p{\TabulatorVZ}@{}p{\TabulatorTX+\TabulatorEH+2\tabcolsep}@{}}
a									& Skalar \\
\textbf{a}				& Tensor 1. Stufe (Vektor)
\end{longtable}

\section*{Lateinische Buchstaben}

\begin{longtable}{@{}p{\TabulatorVZ}@{}p{\TabulatorTX}p{\TabulatorEH}@{}}
	$A$					& Fläche					& $\si{\metre\squared}$ \\
	$C$					& Capazität					& $\si{\ampere\s}$					\\
	$c$					& Konzentration				& $\si{\mole\per\metre\cubed}$\\
	$D$					& Diffusionskonstante		& $\si{\metre\squared\per\second}$\\
	$E$					& Elastizitätsmodul			& $\si{\pascal}$	\\
	$F$					& Kraft						& $\si{\newton}$	\\
	$F_{\text{K}}$		& Faraday-Konstante			& $\si{\coulomb\per\mole}$	\\
	$f_{\pm}$			& Aktivitätskoeffizient		& $\si{\coulomb\per\mole}$	\\
	$G$					& Schubmodul				& $\si{\pascal}$	\\
	$h$					& Plattendicke				& $\si{\metre}$		\\
	$\textbf{i}$		& Stromdichte				& $\si{\ampere\per\metre\squared}$		\\
	$j$					& molare Ionenflussdichte	& $\si{\mole\per\metre\squared\per\second}$		\\
	$K$					& Konstante					& -					\\
	$R_{\text{K}}$		& Unverselle-Gaskonstante	& $\si{\joule\per\mole\per\kelvin}$	\\
	$U_{\theta}$ 		& Elektrochemisches Standardpotenzial	& $\si{\volt}$ \\
	$V$					& Volumen					& $\si{\cubic\metre}$
\end{longtable}

\section*{Griechische Buchstaben}
\begin{longtable}{@{}p{\TabulatorVZ}@{}p{\TabulatorTX}p{\TabulatorEH}@{}}
	$\varepsilon$			& Dehnung						& -	\\
	$\vartheta$				& Temperatur					& $\si{\celsius}$; $\si{\kelvin}$ \\
	$\sigma$				& elektrische Leitfähigkeit		& $\si{\pascal}$ \\
	$\boldsymbol{\sigma}$	& Mechanischer Spannungstensor	& $\si{\pascal}$ \\
	$\sigma_{\text{B,K}}$	& Boltzmann"=Konstante 			& $\si{\joule\per\kelvin}$ \\
	$\nu$					& Poissonzahl					& -	\\
	$\rho$					& Dichte						& $\si{\kilo\per\metre\cubed}$
\end{longtable}

\section*{Indizes, Exponenten und mathematische Akzente}

\begin{longtable}{@{}p{\TabulatorVZ}@{}p{\TabulatorTX+\TabulatorEH+2\tabcolsep}@{}}
	$x_{\text{AM}}$				& Aktivmaterial\\
	$x_{\text{b}}$				& Binder\\
	$x_{\text{DS}}$				& Deckschicht\\
	$x_{\text{echem}}$			& elektro"=chemisch\\
	$x_{\text{exp}}$			& experimentell\\
	$x_{\text{l}}$				& leitende Phase\\
	$x_{\text{s}}$				& specihernde Phase\\
	$x_{\text{mech}}$			& mechanisch\\
	$x_n$						& Normal zur Oberfläche	\\
	$x_{-}$						& negative Elektrode \\
	$x_{\text{OCV}}$			& Gleichgewichtsspannung (\textit{engl.} open-circuit voltage) \\
	$x^{+}$						& positive Elektrode \\
	$x_{\text{th}}$				& thermisch \\
	$x_{\text{theor}}$			& theoretisch \\
	$\vec{x}^T$					& Transponierter Vektor	\\
	$\tilde{x}$				    & Effektivwert 	\\
	$\avrg{x}$					& Mittelwert
\end{longtable}

} % Ende arraystrecth


\clearpage

\cleardoublepage % sonst erhält das Tabellenverzeichnis schon die erste arabische Seitenzahl...
\pagenumbering{arabic} % ab hier Arabische Zahlen für den Hauptteil

%***************************************************************************************************
%********************  EINLEITUNG  *****************************************************************
%***************************************************************************************************
\chapter[Einleitung]{\label{sec:Einleitung}Einleitung}
% max 1 Seite

Die 

%***************************************************************************************************
%********************  Motivation und Zielstellung  ************************************************
%***************************************************************************************************
\section{\label{sec:Motivation_Zielstellung}Problemstellung und Zielsetzung}
% rund 1,5 Seiten inklusive Bild

Die 

%***************************************************************************************************
%********************  LITERATUR�BERSICHT  *********************************************************
%***************************************************************************************************
\section{\label{sec:Stand_Forschung_Technik}Literaturübersicht}
% 3-5 Seiten

Die









   


\chapter{Kapitel}

\section{Unterkapitel}

\section{Unterunterkapitel}
Erstellung Bild siehe Kommentar in .tex Datei
%Bild in Inkscape erzeugt und als SVG sowie pdf_tex speichern (Speichern unter -> .pdf -> Text in PDF weglassen und LaTex Datei erstellen). 

\begin{figure}[h]
	%\raggedleft
		%\def\svgwidth{\columnwidth}
	\def\svgscale{0.98}
		\input{testbild.pdf_tex} 
		\caption{\label{fig:testbild}Testbild erzeugt mit Inkscape}
\end{figure}

Bild \ref{fig:testbild} \cite{Dannemann.Kucher_et.al_AppliedSciences_2018}  

Verwendung Package SIUNITX %(siehe Datei latex_package_readme_siunitx.pdf)   

Anzugsdrehmoment von $M_{\textnormal{a}}=\SI{1.1}{\newtonmetre}$

von \SI{1}{\kilo\hertz} bis \SI{15}{\kilo\hertz}

mittlere Temperaturänderung von $\left\langle \Delta T_{\textnormal{p}}\right\rangle(t)<\SI{0.5}{\degreeCelsius}$

Masse von $m_{\textnormal{p}}=\SI[separate-uncertainty]{0.884 (15)}{\gram}$

(siehe Abschnitt \ref{ch:anhang})
\chapter{Kapitel}

\section{Unterkapitel}

\section{Unterunterkapitel}
\chapter{Kapitel}

\section{Unterkapitel}

\section{Unterunterkapitel}
\chapter{Kapitel}

\section{Unterkapitel}

\section{Unterunterkapitel}
\chapter{Abschlie�ende Bemerkungen}

\section{Zusammenfassung und Bewertung}

Die 

\section{Ausblick}

Die

%Literaturverzeichnis und Datenbank einfügen
\nocite{} %\nocite{*} --> alle in der Datenbank existierenden Einträge werden bearbeitet; ohne * --> nur die verwendeten werden aufgeführt
\bibliographystyle {plaindin_mod} %Aussehen des Literaturverzeichnisses
\bibliography{Literatur_Diss} % Einbinden der Literaturdatenbank <yyyymmdd_Literatur.bib>

\appendix
\chapter{\label{ch:anhang} Anhang}





\cleardoublepage
\pagestyle{empty}
\section*{Lebenslauf}

\begin{tabbing}
\hspace*{4cm}\= \kill

\textbf{Persönliche Daten}\\[0.45em]%
Name							\>  Willi Zschiebsch	\\[0.45em]
Geburtsdatum					\> 	28.10.1996			\\[0.45em]
Geburtsort						\> 	Wurzen  			\\[0.45em]
Eltern							\> 						\\
								\> 						\\[0.45em]
Familienstand					\> 	ledig				\\[0.45em]
Kinder      					\> 	keine				\\[0.45em]
            					\> 						\\[0.45em]

Staatsangehörigkeit		\> 		deutsch					\\[1.5em]

\textbf{Ausbildung} \\[0.45em]
2015 - 2018	\> Maschinenbau Master an der HTWK Leipzig	\\[0.5em]	
2018	\> Forschungsaufenthalt im Robotiklabor von Prof. Amir Shapiro 	\\[0.5em]
2015 - 2018	\> Maschinenbau Bachelor an der HTWK Leipzig	\\[0.5em]
2015	\> Abitur Wilhelm-Ostwaldgymnasium Leipzig	\\[0.5em]
															\\[1.5em]

\textbf{Auszeichnungen und Stipendien} \\[0.45em]
03/2015 \> Sonderpreis des Ministeriums für Wirtschaft, Arbeit und Verkehr, Sachsen \\[0.5em]
06/2015 \>  Preis für eine besondere Leistung im Gebiet der Technik von Heinz und \\ 
		\> Gisela Friederichs Stiftung \\[0.5em]
06/2015 \>  Qualifizierung und Teilnahme für den Bundeswettbewerbes „jugendForscht“ \\[0.5em]
12/2015–12/2020 \>  Mitglied in der Studienstiftung des deutschen Volkes \\[0.5em]
10/2018 \> Preis der Karl-Kolle-Stiftung \\[0.5em]
06/2019 \> Dritter Platz des VDI-Förderpreises \\[0.5em]
10/2021 \> Preis der Karl-Kolle-Stiftung \\[0.5em]
06/2021-05/2024 \>  HTWK-Promotionsstipendium \\[0.5em]
04/2022 \> Best Speaker-Award auf der ICOMS21 \\[0.5em]
09/2022 \> Best Presentation Award auf der ICAMDS 2023 \\[0.5em]

\end{tabbing}





%%Diplom-\\
%%praktikum:\> 1994\> Institut f"ur Theoretische Physik\\
%%\>\>(Abt. B) der TU Clausthal\\
%wiss. Hilfskraft:\> 1993 - 1994\> Physikalisches Institut (Abt. Experimentalphysik)\\
%\>\> der TU Clausthal\\
%\> 1995\>  Institut f"ur Theoretische Physik\\
%\>\> (Abt. B) der TU Clausthal\\
%\> 1996 - 1998\> Institut f"ur Leichtbau und Kunststofftechnik\\
%\>\> der TU Dresden\\
%Stipendien:\> 1997 - 1998\> Landesinnovationsstipendium Sachsen (LIST)\\
%wiss. Mitarbeiter:\> seit 1998\> Institut f"ur Leichtbau und Kunststofftechnik\\
%\>\> der TU Dresden
%\end{tabbing}

\end{document}

